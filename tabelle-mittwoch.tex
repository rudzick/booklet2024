%\newpage
%\enlargethispage{0.0\baselineskip}
\renewcommand{\arraystretch}{1.4}
\section*{Vorträge am Mittwoch}\label{mittwoch}
\renewcommand{\conferenceDay}{\mittwoch}
\setPageBackground
\noindent\begin{tabular}{Z{0.7cm}Z{6.85cm}}
  & \multicolumn{1}{c}{\cellcolor{geoblau} Hörsaal 1 (0\textquotesingle 115)}
  \tabularnewline
  10:00
  \talk{Eröffnung}{FOSSGIS e.V.}
  \tabularnewline
 10:50
  \talk{Keynote LGV HH}{Thomas Eichhorn}
  \tabularnewline
  11:15
  \talk{Sovereign Cloud Stack (SCS): Offene, förderierbare Cloud-Technologie für jeden Sektor}{Manuela Urban}
  \tabularnewline
\end{tabular}

\noindent\begin{tabular}{Z{0.7cm}Z{3.0cm}Z{3.0cm}}
  & \multicolumn{1}{c}{\cellcolor{geoblau} HS 1 (Audimax I)}
  & \multicolumn{1}{c}{\cellcolor{hellgelb} HS 2 (Dietze H016)}
  \tabularnewline
11:45
  \talk{Automatisierte Bestimmung der Straßenbeschaffenheit mit Machine Learning}{Alexandra Kapp, Edith Hoffmann}
  \talk{Neues in STAC}{Matthias Mohr}
  \tabularnewline
  12:20
  \talk{Der Weg eines Schlaglochs von der Straße auf die Karte}{
Asmus Harder, Melanie Fleischer}
  \talk{Bereitstellung von freien Geodaten (OpenData) mit STAC beim LGLN}{
Katrin Pinkert, Ralf Wohlfahrt}
  \tabularnewline
%  \rowcolor{commongray}
 % 12:45 & \multicolumn{2}{c}{%
 %   \parbox[c]{24pt}{%
 %     \includegraphics[height=10pt]{restaurant}%
 %   }
 %   Mittagspause
%  } \tabularnewline
\end{tabular}
%
%\vspace{0.5\baselineskip}

\noindent\begin{tabular}{Z{0.7cm}Z{3.0cm}Z{3.0cm}}
  & \multicolumn{1}{c}{\cellcolor{hellgruen} HS 3 (K0506/ Audimax II)}
  & \multicolumn{1}{c}{\cellcolor{dezentrot} HS 4 (A.013)}
  \tabularnewline
  11:45
  \talk{Stand des GRASS GIS Projekts: Nicht was Sie denken!}{Markus Neteler}
  \talk{Mapbender - die neue Version 4 stellt sich vor}{Astrid Emde}
  \tabularnewline
  12:20
  \talk{Wie schreibe ich ein GRASS GIS Addon?}{Markus Neteler, Carmen Tawalika}
  \talk{PyQGIS Schnuppervortrag – Mein erstes Plugin für QGIS}{Gordon Schlolaut}
  \tabularnewline
  \rowcolor{commongray}
  12:45 & \multicolumn{2}{c}{%
    \parbox[c]{24pt}{%
      \includegraphics[height=10pt]{restaurant}%
    }
    Mittagspause
  } \tabularnewline
\end{tabular}

\noindent\begin{tabular}{Z{0.7cm}Z{3.0cm}Z{3.0cm}}
  & \multicolumn{1}{c}{\cellcolor{geoblau} HS 1 (0\textquotesingle 115)}
  & \multicolumn{1}{c}{\cellcolor{hellgelb} HS 2 (0\textquotesingle 110)}
\tabularnewline
  14:15
  \talk{Open Data @Greenpeace e.V.}{Jonathan Niesel}
  \talk{BBOX: Kompakter OGC API Server für Features, Tiles und mehr}{Pirmin Kalberer}
 \tabularnewline
  14:50
  \talk{Hinweiskarten Starkregengefahren: OpenData für die bundesweite Klimawandelanpassung}{Lukas Wimmer}
\talk{pygeoapi - eine Python Server Software für OGC API Standards}{Astrid Emde}
  \tabularnewline
  15:25
 \talk{Starkregengefahrenhinweiskarten für Niedersachsen/Schleswig-Holstein/HB und Hamburg}{Barbara Werth, Uwe Ross}
  \talk{pg\_featureserv - Veröffentlichung von Vektordaten mit OGC API Features}{Jakob Miksch}
    \tabularnewline
  \rowcolor{commongray}
  16:00 & \multicolumn{2}{c}{%
    \parbox[c]{24pt}{%
      \includegraphics[height=10pt]{cafe}%
    }
    Kaffeepause \emph{(30 min)}}
   \tabularnewline
\end{tabular}

\renewcommand{\arraystretch}{1.4}
\noindent\begin{tabular}{Z{0.7cm}Z{3.0cm}Z{3.0cm}}
  & \multicolumn{1}{c}{\cellcolor{hellgruen} HS 3 (K0506/ Audimax II)}
  & \multicolumn{1}{c}{\cellcolor{dezentrot} HS 4 (A.013)}
  \tabularnewline
  14:15
  \talk{Aufbau einer Agrar-Forschungsdateninfra\-struktur mit GeoNode und Kubernetes}{Marcel Wallschläger}
  \talk{xPlanBox: kommunale Datendrehscheibe für die Bauleit- und Landschaftsplanung}{Stefan Peuser}
  \tabularnewline
  14:50
  \talk{GeoNetwork-UI: Ein anwenderfreundliches Frontend für den Datenkatalog GeoNetwork}{Angelika Kinas}
  \talk{XPlanung für die Cloud}{Torsten Friebe}
    \tabularnewline
  15:25
  \talk{GeoServer Cloud mit Kubernetes}{Nils Bühner}
  \talk{Einsatz von Machine Learning zur Erstellung von XPlanGML}{Julian Zilz}
  \tabularnewline
%  \rowcolor{commongray}
%  16:00 & \multicolumn{2}{c}{%
%   \parbox[c]{24pt}{%
%      \includegraphics[height=10pt]{cafe}%
%   }
%    Kaffeepause
% } \tabularnewline
\end{tabular}

%\subsubsection*{Anwendertreffen und Demo-Sessions am Mittwoch}%\label{demomittwoch}
\label{demomittwoch}
\renewcommand{\arraystretch}{1.4}
   \enlargethispage{10\baselineskip}
\justifying\setPageBackground

\noindent\begin{tabular}{Z{0.7cm}Z{2.0cm}Z{2.0cm}Z{2.0cm}}
  & \multicolumn{1}{c}{\cellcolor{audimax}\small Exp (H.04)}
  & \multicolumn{1}{c}{\cellcolor{eins}\small Anw\,/\,BoF\,1\,(H.09)}
  & \multicolumn{1}{c}{\cellcolor{zwei}\small Anw\,/\,BoF\,2\,(H.08)}
  \tabularnewline
14:15
  \talk{Lizmap Webclient \emph{(60min)}}{Günter Wagner}
  \talk{Studierende stellen Ihre Arbeit vor \emph{(1h30min)}}{}
  \talk{}{}
  \tabularnewline
%  \rowcolor{commongray}
% 16:00 & \multicolumn{3}{c}{%
 %   \parbox[c]{24pt}{%
 %     \includegraphics[height=10pt]{cafe}%
%    }
%    Kaffeepause
%  } \tabularnewline
  \end{tabular}

\noindent\begin{tabular}{Z{0.7cm}Z{3.0cm}Z{3.0cm}}
  & \multicolumn{1}{c}{\cellcolor{geoblau} HS 1 (0\textquotesingle 115)}
  & \multicolumn{1}{c}{\cellcolor{hellgelb} HS 2 (0\textquotesingle 110)}
\tabularnewline
  16:30
     \talk{Lightning-Talks}{}
  \talk{Weltwärmestrom Datenbank Projekt: webbasierte Explorationswerkzeuge für Punktdaten}{Nikolas Ott}
  \tabularnewline
  17:05
  \talk{Es ist doch nur Software - Zusammenfassung der AG-Aktivitäten}{Torsten Friebe, Torsten Wiebke, Florian Micklich, David Arndt}
  \talk{Monitoring von Waldgebieten mit Hilfe von Sentinel-2 abgeleiteten Vegetationsidizes}{Markus Eichhorn}
    \tabularnewline
  17:40
  \longTalk{2}{Nachhaltige Beschaffung mit Blick auf Open Source}{Miriam Seyffarth, Florian Micklich, David Arndt}
  \talk{Open Source and Web-Based GeoAI tool for Transparent Forest Fire Prediction}{Sebastian Meier}
  \tabularnewline
  18:15
  \talk{}{}
\talk{QGIS Plugin: What's the impact of my installed dam on the vegetation around it?}{Berit Mohr}
   \tabularnewline
\end{tabular}
\normalsize
\newpage

\vspace{0.5\baselineskip}
\enlargethispage{4.0\baselineskip}
\renewcommand{\arraystretch}{1.4}
\renewcommand{\baselinestretch}{1.1}

\vspace{0.5\baselineskip}
%\enlargethispage{1.0\baselineskip}

\noindent\begin{tabular}{Z{0.7cm}Z{3.0cm}Z{3.0cm}}
  & \multicolumn{1}{c}{\cellcolor{hellgruen} HS 3 (K0506/ Audimax II)}
  & \multicolumn{1}{c}{\cellcolor{dezentrot} HS 4 (A.013)}
  \tabularnewline
  16:30
  \talk{POLAR - Vollkonfigurierbare, pluginbasierte Kartenklienten für bürgernahe Anwendungen}{P. Röhling, D. Sen}
  \talk{Modellierung von Fuzzyness \/ Wobbliness in Geodaten}{Florian Thiery}
  \tabularnewline
  17:05
%  \talk{DIPAS - Digitale Bürgerbeteiligung mit Open Source \& Open Data}{Daniel Bockelmann, Mateusz Lendziński}
  \talk{DIPAS - Digitale Bürgerbeteiligung mit Open Source \& Open Data}{D. Bockelmann, M. Lendziński}
  \talk{Geodaten mit DuckDB verarbeiten}{Jakob Miksch, Nikolai Janakiev}
  \tabularnewline
  17:40
  \talk{Ein WebGIS zur Öffentlichkeitsbeteiligung in Planungsverfahren des Netzausbaus}{Kim-Jana Stückemann}
%  \talk{The SPARQL Unicorn Ontology documentation: Exposing RDF geodata using static GeoAPIs}{Florian Thiery, Timo Homburg}
  \talk{The SPARQL Unicorn Ontology documentation: Exposing RDF geodata using static GeoAPIs}{F. Thiery, T. Homburg}
  \tabularnewline
 18:15
  \talk{Lightning-Talks}{}
  \talk{INSPIRE2GPKG}{Armin Retterath}
 %\tabularnewline
  \end{tabular}
  
\label{endevortraegemittwoch}
\renewcommand{\arraystretch}{1.4}
%\justifying\setPageBackground

\noindent\begin{tabular}{Z{0.7cm}Z{2.0cm}Z{2.0cm}Z{2.0cm}}
  & \multicolumn{1}{c}{\cellcolor{audimax}\small Exp (H.04)}
  & \multicolumn{1}{c}{\cellcolor{eins}\small Anw\,/\,BoF\,1\,(H.09)}
  & \multicolumn{1}{c}{\cellcolor{zwei}\small Anw\,/\,BoF\,2\,(H.08)}
  \tabularnewline
16:30
%  \talk{Ask me anything QGIS! \emph{(60 min)}}{Marco Bernasocchi, Matthias Kuhn}
  \talk{Ask me anything QGIS! \emph{(60 min)}}{M. Bernasocchi, M. Kuhn}
  \talk{Studierende stellen Ihre Arbeit vor \emph{(1h30min)}}{}
  \talk{Anwendertreffen Lizmap-Webclient \emph{(60min)}}{Günter Wagner}
\tabularnewline
  \end{tabular}
  
\newpage
