
% time: Friday 09:00
% URL: https://pretalx.com/fossgis2023/talk/fossgis2024-38766-indoor-osm/

%
\newTimeslot{09:00}
\noindent\abstractOther{%
  Tobias Knerr, Volker Krause%
}{%
  Indoor OSM%
}{%
}{%
  Ein Treffen für alle, die als Mapper oder Entwickler mit Indoor-Karten in OpenStreetMap zu tun
  haben%
}%
{%
  Anwendertreffen/BoF1 (H.09)%
}%



%%%%%%%%%%%%%%%%%%%%%%%%%%%%%%%%%%%%%%%%%%%

% time: Friday 09:00
% URL: https://pretalx.com/fossgis2023/talk/fossgis2024-38601-transformationspotenziale-groflchiger-parkpltze-fr-den-nachhaltigen-stadtumbau/

%

\noindent\abstractOther{%
  Anika Weinmann, Vanessa Dunker, Max Bohnet, Julia Haas%
}{%
  Transformationspotenziale großflächiger Parkplätze für den nachhaltigen Stadtumbau%
}{%
}{%
  Angesichts der Anforderungen an einen nachhaltigen zukunftsweisenden Stadtumbau werden im Rahmen
  des Forschungsprojekts Transformationspotenziale großflächiger Parkplätze untersucht. Dafür werden
  diese Flächen unter der Nutzung von Open Data und Open-Source-GIS-Software methodisch erfasst,
  klassifiziert und bewertet sowie Transformationspotenziale aufgezeigt. Zusätzlich verdeutlichen
  Best-Practice-Beispiele, wie die Transformation großflächiger Parkplätze gelingen kann.%
}%
{%
  Hörsaal 1 (Audimax 1)%
}%



%%%%%%%%%%%%%%%%%%%%%%%%%%%%%%%%%%%%%%%%%%%

% time: Friday 09:00
% URL: https://pretalx.com/fossgis2023/talk/fossgis2024-38918-modulare-foss-dateninfrastrukturen-fr-kommunen/

%

\noindent\abstractOther{%
  Prof. Dr. Sebastian Meier%
}{%
  Modulare FOSS Dateninfrastrukturen für Kommunen%
}{%
}{%
  Im Bereich der Bereitstellung (offener) Daten durch die öffentliche Verwaltung, gibt es bereits
  eine ganze Bandbreite etablierter FOSS-Anwendungen. Wir stellen in unserem Vortrag eine modulare
  Plattform für die Datenhaltung und -bereitstellung vor, welche sich aus verschiedenen FOSS-Modulen
  zusammensetzt. Statt neue Software zu entwickeln, wollen wir aufzeigen, wie bestehende
  Anwendungen, welche von verschiedenen Communities entwickelt werden, verknüpft werden können, um
  daraus eine integrative%
}%
{%
  Hörsaal 2 (Dietze H016)%
}%



%%%%%%%%%%%%%%%%%%%%%%%%%%%%%%%%%%%%%%%%%%%

% time: Friday 09:00
% URL: https://pretalx.com/fossgis2023/talk/fossgis2024-39048-neue-geoperspektiven-nach-10-jahren-digital-souverne-softwareentwicklung-am-bfs/

%

\noindent\abstractOther{%
  Dr. Marco Lechner%
}{%
  Neue Geoperspektiven nach 10 Jahren digital souveräne Softwareentwicklung am BfS%
}{%
}{%
  Seit nunmehr 10 Jahren entwickelt das Bundesamt für Strahlenschutz seine Notfallschutzsysteme
  gemäß einer Entwicklungsstrategie, die digitale Souveränität und Nachhaltigkeit sichert, zu
  verschiedenen Open Source GIS Projekten Beiträge leisten konnte oder solche selbst entwickelt und
  veröffentlicht hat. Der Vortrag soll die gemachten Erfahrungen, Erfolgs- und Irrwege,
  Ausschreibungsstrategien, bis hin zu Betriebskonzepten beleuchten und dabei anregen mehr digitale
  Souveränität zu wagen.%
}%
{%
  Hörsaal 3 (K0506/ Audimax II)%
}%



%%%%%%%%%%%%%%%%%%%%%%%%%%%%%%%%%%%%%%%%%%%

% time: Friday 09:00
% URL: https://pretalx.com/fossgis2023/talk/fossgis2024-38880-ohsomenow-osm-daten-in-echtzeit-analysieren/

%

\noindent\abstractOther{%
  Benjamin Herfort%
}{%
  ohsomeNow: OSM-Daten in Echtzeit analysieren%
}{%
}{%
  Wir wollen "`ohsomeNow"' vorstellen und zeigen wie wir den Datenstrom, den die OSM-User minütlich
  produzieren, effizient analysieren und visualisieren können.%
}%
{%
  Hörsaal 4 (A.013)%
}%



%%%%%%%%%%%%%%%%%%%%%%%%%%%%%%%%%%%%%%%%%%%

% time: Friday 09:05
% URL: https://pretalx.com/fossgis2023/talk/fossgis2024-38910-osm-mapathon-auf-der-konferenz-der-geodsiestudierenden/

%
\newTimeslot{09:05}
\noindent\abstractOther{%
  Florian Thiery, Nicole Habersack%
}{%
  OSM Mapathon auf der Konferenz der GeodäsieStudierenden%
}{%
}{%
  Open Data und Open Source Software (z.B. Open Street Map und QGIS) gewinnt im Geodäsie-Studium
  stetig an Bedeutung. Zum Vergrößern dieses Datenspeichers werden immer öfter Mapathons
  veranstaltet, z.B. über "`Missing Maps”. Dieser Lightning Talk gibt einen kurzen Einblick in die
  Mapathons der KonGeoS Karlsruhe und Oldenburg (beide 2023). Der Vortrag soll zudem zur Diskussion
  anregen, wie die OSM-Community und die KonGeoS zukünftig enger zusammenarbeiten können.%
}%
{%
  Hörsaal 4 (A.013)%
}%



%%%%%%%%%%%%%%%%%%%%%%%%%%%%%%%%%%%%%%%%%%%

% time: Friday 09:10
% URL: https://pretalx.com/fossgis2023/talk/fossgis2024-39701-spontaner-lt/

%
\newTimeslot{09:10}
\noindent\abstractOther{%
  %
}{%
  Spontaner LT%
}{%
}{%
  Spontan eingereichte Lightning Talks. Jeder kann einen Vortrag halten, Registrierung erfolgt an
  einer entsprechenden Pinnwand im Foyer.%
}%
{%
  Hörsaal 4 (A.013)%
}%



%%%%%%%%%%%%%%%%%%%%%%%%%%%%%%%%%%%%%%%%%%%

% time: Friday 09:15
% URL: https://pretalx.com/fossgis2023/talk/fossgis2024-39702-spontaner-lt/

%
\newTimeslot{09:15}
\noindent\abstractOther{%
  %
}{%
  Spontaner LT%
}{%
}{%
  Spontan eingereichte Lightning Talks. Jeder kann einen Vortrag halten, Registrierung erfolgt an
  einer entsprechenden Pinnwand im Foyer.%
}%
{%
  Hörsaal 4 (A.013)%
}%



%%%%%%%%%%%%%%%%%%%%%%%%%%%%%%%%%%%%%%%%%%%

% time: Friday 09:35
% URL: https://pretalx.com/fossgis2023/talk/fossgis2024-38987-mit-fahrradstndern-und-parkbnken-fr-die-verkehrswende/

%
\newTimeslot{09:35}
\noindent\abstractOther{%
  Tobias Jordans%
}{%
  Mit Fahrradständern und Parkbänken für die Verkehrswende%
}{%
}{%
  Gute Infrastruktur zum Parken von Fahrrädern ist ein wichtiges Puzzlestück der Verkehrswende,
  insbesondere für intermodale Angebote. Die Bedeutung von guten Rastmöglichkeiten auf Gehwegen
  steigt mit jedem Jahr, in dem unsere Gesellschaft altert. In OpenStreetMap sammeln wir beide
  Datenklassen. Und im radverkehrsatlas.de machen wir sie für ganz Deutschland und die Verkehrswende
  einfach zugänglich.%
}%
{%
  Hörsaal 1 (Audimax 1)%
}%



%%%%%%%%%%%%%%%%%%%%%%%%%%%%%%%%%%%%%%%%%%%

% time: Friday 09:35
% URL: https://pretalx.com/fossgis2023/talk/fossgis2024-38569-alles-fit-praxiserfahrungen-mit-geohealthcheck/

%

\noindent\abstractOther{%
  Oliver Schmidt%
}{%
  Alles fit?~-- Praxiserfahrungen mit GeoHealthCheck%
}{%
}{%
  Die unabhängige Überwachung der bereitgestellten Geodatendienste ist für anbietende Stellen von
  großer Bedeutung. Hierfür bietet sich GeoHealthCheck an, das auf Python basiert und auch als
  Docker-Anwendung existiert. Praxiserfahrungen und Beispiele aus der täglichen Anwendung werden in
  diesem Vortrag vorgestellt.%
}%
{%
  Hörsaal 2 (Dietze H016)%
}%



%%%%%%%%%%%%%%%%%%%%%%%%%%%%%%%%%%%%%%%%%%%

% time: Friday 09:35
% URL: https://pretalx.com/fossgis2023/talk/fossgis2024-38904-geodateninfrastruktur-step-by-step-von-proprietrer-zu-offener-software/

%

\noindent\abstractOther{%
  Stefan Peuser%
}{%
  Geodateninfrastruktur: Step by step von proprietärer zu offener Software%
}{%
}{%
  Das KRZN stellt den Kommunen am Niederrhein eine kommunale Geodateninfrastruktur (GDI) bereit und
  entwickelt diese kontinuierlich weiter. Seit einiger Zeit kommen vermehrt OpenSource-Verfahren zum
  Einsatz. Wenngleich diverse Hindernisse den Einsatz von (mehr) OpenSource aktuell noch erschweren
  oder gar verhindern, bestärken eine Reihe von Erfolgsgeschichten das KRZN, den Weg in Richtung
  OpenSource im GDI-Kontext in Zusammenarbeit mit kommunalen Akteuren:innen weiterzugehen.%
}%
{%
  Hörsaal 3 (K0506/ Audimax II)%
}%



%%%%%%%%%%%%%%%%%%%%%%%%%%%%%%%%%%%%%%%%%%%

% time: Friday 09:35
% URL: https://pretalx.com/fossgis2023/talk/fossgis2024-38978-ein-plugin-mit-direktem-zugang-zu-den-ressourcen-aus-dem-qgis-hub/

%

\noindent\abstractOther{%
  Andreas Jobst%
}{%
  Ein Plugin mit direktem Zugang zu den Ressourcen aus dem QGIS Hub%
}{%
}{%
  Mit dem QGIS Hub Plugin stellen wir einen neuen Plugin vor, der es Nutzern ermöglicht, die stetig
  wachsenden Ressourcen aus dem Hub direkt aus QGIS heraus zu erkunden und in ihr jeweiliges Projekt
  einzubinden.%
}%
{%
  Hörsaal 4 (A.013)%
}%



%%%%%%%%%%%%%%%%%%%%%%%%%%%%%%%%%%%%%%%%%%%

% time: Friday 09:40
% URL: https://pretalx.com/fossgis2023/talk/fossgis2024-38536-jetzt-neu-event-driven-enterprise-data-cloud-api-/

%
\newTimeslot{09:40}
\noindent\abstractOther{%
  Arnulf Benno Christl%
}{%
  Jetzt Neu: Event Driven Enterprise Data Cloud API!%
}{%
}{%
  Jetzt Neu: Event Driven Enterprise Data Cloud API! Zu Risiken und Nebenwirkungen fragen Sie Ihren
  DevTestSecDingsOps oder Cloud Anbieter.%
}%
{%
  Hörsaal 4 (A.013)%
}%



%%%%%%%%%%%%%%%%%%%%%%%%%%%%%%%%%%%%%%%%%%%

% time: Friday 09:45
% URL: https://pretalx.com/fossgis2023/talk/fossgis2024-38985-road-network-contraction-mit-postgis/

%
\newTimeslot{09:45}
\noindent\abstractOther{%
  Felix Sommer%
}{%
  Road Network Contraction mit PostGIS%
}{%
}{%
  Bei der Berechnung des kürzesten Weges zwischen zwei Orten ist die Größe des Straßendatensatzes,
  bezogen auf die Performanz, oft der wichtigste Faktor.
  Die Open Source Erweiterung PostGIS bietet Tools an, um Datensätze mit Strassen-Netzwerken zu
  minimieren. Dieses Verfahren, auch bekannt als "`Road Network Contraction”, wird in dieser
  Präsentation vorgestellt, sowie unterschiedliche Möglichkeiten dieses Verfahren zu verfeinern.%
}%
{%
  Hörsaal 4 (A.013)%
}%



%%%%%%%%%%%%%%%%%%%%%%%%%%%%%%%%%%%%%%%%%%%

% time: Friday 09:50
% URL: https://pretalx.com/fossgis2023/talk/fossgis2024-39699-spontaner-lt/

%
\newTimeslot{09:50}
\noindent\abstractOther{%
  %
}{%
  Spontaner LT%
}{%
}{%
  Spontaner Lightning Talk.
  Eintragung siehe Liste Liste im Foyer.Eintragung siehe Liste Liste im Foyer.Eintragung siehe Liste
  Liste im Foyer.%
}%
{%
  Hörsaal 4 (A.013)%
}%



%%%%%%%%%%%%%%%%%%%%%%%%%%%%%%%%%%%%%%%%%%%

% time: Friday 10:10
% URL: https://pretalx.com/fossgis2023/talk/fossgis2024-39003-radnetz-qualitt-mit-openstreetmap-daten-auswerten/

%
\newTimeslot{10:10}
\noindent\abstractOther{%
  Alex Seidel%
}{%
  Radnetz-Qualität mit OpenStreetMap-Daten auswerten%
}{%
}{%
  Mit einem OSM-basierten Radverkehrs-Qualitätsindex möchten wir eine niedrigschwellige Methode zur
  Analyse von Radnetzen bereitstellen. Wir geben Einblicke in den Proof Of Concept aus Berlin, wo
  wir detaillierte OSM-Daten zur Bewertung der Radinfrastruktur erhoben und ausgewertet haben. Ein
  solcher Index macht Lücken im Netz und somit Handlungsbedarf für die Verkehrsplanung sichtbar und
  zeigt, wie klein der Bewegungsradius für vulnerable Gruppen wie Kinder auf dem Fahrrad zum Teil
  ist.%
}%
{%
  Hörsaal 1 (Audimax 1)%
}%



%%%%%%%%%%%%%%%%%%%%%%%%%%%%%%%%%%%%%%%%%%%

% time: Friday 10:10
% URL: https://pretalx.com/fossgis2023/talk/fossgis2024-38811-interaktive-dashboards-zur-optimierung-von-intelligence-prozessen/

%

\noindent\abstractOther{%
  Jan Suleiman, Hannes Blitza%
}{%
  Interaktive Dashboards zur Optimierung von Intelligence Prozessen%
}{%
}{%
  Durch stetig wachsende Mengen verfügbarer Daten, u.a. getrieben durch Open Data Policies, gewinnen
  Business Intelligence (BI) Tools zunehmend an Bedeutung. Nicht nur Unternehmen nutzen Dashboards
  zu Analyse und Visualisierung, ebenso können öffentliche Verwaltungen derlei Tools für
  demokratische Entscheidungsprozesse, sowie Bürgerpartizipation verwenden. Dieser Vortrag zeigt,
  wie private und öffentliche Akteure ihre Geodaten in BI-Prozessen mithilfe von Apache Superset
  einsetzen können.%
}%
{%
  Hörsaal 2 (Dietze H016)%
}%



%%%%%%%%%%%%%%%%%%%%%%%%%%%%%%%%%%%%%%%%%%%

% time: Friday 10:10
% URL: https://pretalx.com/fossgis2023/talk/fossgis2024-39055--switch2qgis-komplettablsung-proprietrer-gi-systeme-mit-qgis-langzeiterfahrungen/

%

\noindent\abstractOther{%
  Mike Elstermann%
}{%
  \#switch2qgis: Komplettablösung proprietärer GI-Systeme mit QGIS~-- Langzeiterfahrungen%
}{%
}{%
  Der Vortrag beschreibt den kompletten Übergang der Ablösung von seit mehr als 20 Jahren
  etablierten proprietären GI-Systeme und der Geodatenbanken des Marktführers durch den
  vollständigen Ersatz durch OSS, insbesondere das freie QGIS in der Stadtverwaltung Halle (Saale)
  und die langjährigen Erfahrungen mit dieser Umstellung.%
}%
{%
  Hörsaal 3 (K0506/ Audimax II)%
}%



%%%%%%%%%%%%%%%%%%%%%%%%%%%%%%%%%%%%%%%%%%%

% time: Friday 10:10
% URL: https://pretalx.com/fossgis2023/talk/fossgis2024-39053-datenverknpfung-von-befragungsdaten-mit-geodaten-wie-geht-das-/

%

\noindent\abstractOther{%
  Theodor Rieche%
}{%
  Datenverknüpfung von Befragungsdaten mit Geodaten~-- wie geht das?%
}{%
}{%
  Wie lassen sich Befragungs- und Geodaten für die Beforschung interdisziplinärer Forschungsfragen
  technisch verknüpfen? Im Forschungsprojekt "`Aufbau der Sozial-Raumwissenschaftlichen
  Forschungsdateninfrastruktur SoRa: FAIR, intelligent, integrativ (SoRa+)"' wird eine verteilte und
  modulare Infrastruktur entwickelt, welche die Datenverknüpfung durch Dienste unter
  Berücksichtigung von Datenschutz und FAIR-Prinzipien für die Wissenschaft erleichtern soll.%
}%
{%
  Hörsaal 4 (A.013)%
}%



%%%%%%%%%%%%%%%%%%%%%%%%%%%%%%%%%%%%%%%%%%%

% time: Friday 10:15
% URL: https://pretalx.com/fossgis2023/talk/fossgis2024-38723-agiles-forschungsdatenmanagement-mit-linkahead/

%
\newTimeslot{10:15}
\noindent\abstractOther{%
  Thomas Weiß, Daniel Hornung%
}{%
  Agiles Forschungsdatenmanagement mit LinkAhead%
}{%
}{%
  In diesem Lightning Talk stellen wir LinkAhead vor, eine unter AGPL lizensierte
  Datenmanagementsoftware. Sie verfolgt einen flexiblen Ansatz in Sachen Datenmodell und arbeitet im
  Baukastenprinzip~-- die Software lässt sich an jede gewachsene Datenumgebung im Nachhinein
  anpassen. Dabei bleiben bestehende Workflows bestehen~-- Nutzer:innen können weiter arbeiten wie
  bisher und machen ihre Daten dabei nachvoll- und reproduzierbar. Rechte-, User- und
  Versionsmanagement inklusive.%
}%
{%
  Hörsaal 4 (A.013)%
}%



%%%%%%%%%%%%%%%%%%%%%%%%%%%%%%%%%%%%%%%%%%%

% time: Friday 10:20
% URL: https://pretalx.com/fossgis2023/talk/fossgis2024-39703-spontaner-lt/

%
\newTimeslot{10:20}
\noindent\abstractOther{%
  %
}{%
  Spontaner LT%
}{%
}{%
  Spontan eingereichte Lightning Talks. Jeder kann einen Vortrag halten, Registrierung erfolgt an
  einer entsprechenden Pinnwand im Foyer.%
}%
{%
  Hörsaal 4 (A.013)%
}%



%%%%%%%%%%%%%%%%%%%%%%%%%%%%%%%%%%%%%%%%%%%

% time: Friday 10:25
% URL: https://pretalx.com/fossgis2023/talk/fossgis2024-39700-spontaner-lt/

%
\newTimeslot{10:25}
\noindent\abstractOther{%
  %
}{%
  Spontaner LT%
}{%
}{%
  Spontan eingereichte Lightning Talks. Jeder kann einen Vortrag halten, Registrierung erfolgt an
  einer entsprechenden Pinnwand im Foyer.%
}%
{%
  Hörsaal 4 (A.013)%
}%



%%%%%%%%%%%%%%%%%%%%%%%%%%%%%%%%%%%%%%%%%%%

% time: Friday 11:10
% URL: https://pretalx.com/fossgis2023/talk/fossgis2024-38496-qwc2-anwendertreffen/

%
\newTimeslot{11:10}
\noindent\abstractOther{%
  Daniel Cebulla%
}{%
  QWC2 Anwendertreffen%
}{%
}{%
  Der QWC2 (QGIS Web Client 2) ist die offizielle WebGIS-Anwendung des QGIS Projektes.  Das Treffen
  soll QWC2-Anwendern und -Administratoren die Möglichkeit geben, eigene Erfahrungen mit anderen
  Anwendern zu teilen und neue Kontakte zu knüpfen. Teilnehmer können ihre eigenen, mit QWC2
  realisierten WebGIS-Projekte vorstellen und gemeinsam evtl. auftretende Probleme diskutieren oder
  anderen Tipps geben.
  Sie sind herzlich eingeladen, zum Anwendertreffen zu kommen!%
}%
{%
  Anwendertreffen/BoF1 (H.09)%
}%



%%%%%%%%%%%%%%%%%%%%%%%%%%%%%%%%%%%%%%%%%%%

% time: Friday 11:10
% URL: https://pretalx.com/fossgis2023/talk/fossgis2024-38554-wie-vollstndig-sind-die-daten-bei-openstreetmap-/

%

\noindent\abstractOther{%
  Mathias Gröbe%
}{%
  Wie vollständig sind die Daten bei OpenStreetMap?%
}{%
}{%
  Die Daten von OpenStreetMap gelten in Mitteleuropa als sehr vollständig und gut gepflegt~-- aber
  ist das auch wirklich so? Anhand von wie Beispielen Postleitzahlengebiete, Adressen und
  Kilometerangaben an Bahnstrecken wird genauer betrachtet, welche Daten von OpenStreetMap sich gut
  oder besser nicht in Projekten praktisch verwenden lassen.%
}%
{%
  Hörsaal 1 (Audimax 1)%
}%



%%%%%%%%%%%%%%%%%%%%%%%%%%%%%%%%%%%%%%%%%%%

% time: Friday 11:10
% URL: https://pretalx.com/fossgis2023/talk/fossgis2024-38830-eoc-geoservice-datenstze-und-services/

%

\noindent\abstractOther{%
  Felix Feckler%
}{%
  EOC Geoservice~-- Datensätze und Services%
}{%
}{%
  In diesem Vortrag soll ein Überblick über den EOC Geoservice gegeben werden. Ein Schwerpunkt wird
  auf den zur Verfügung gestellten Datensätzen und den Services / Schnittstellen liegen. Bei den
  Schnittstellen wird verstärkt auf STAC ( Spatial Temporal Asset Catalog ) eingegangen, um Filter-
  und Analysemöglichkeiten (z.B. anhand eines JupyterNotebooks) aufzuzeigen.%
}%
{%
  Hörsaal 2 (Dietze H016)%
}%



%%%%%%%%%%%%%%%%%%%%%%%%%%%%%%%%%%%%%%%%%%%

% time: Friday 11:10
% URL: https://pretalx.com/fossgis2023/talk/fossgis2024-38397-openlayers-mehr-als-nur-karten-im-web/

%

\noindent\abstractOther{%
  Andreas Hocevar, Marc Jansen%
}{%
  OpenLayers~-- mehr als nur Karten im Web%
}{%
}{%
  OpenLayers ist die erste Wahl für Karten im Web, wenn umfangreiche Interaktionen mit Karte und
  Daten gewünscht sind. Dieser Vortrag zeigt anhand von Beispielen, wie mit wenig Programmcode auf
  unterschiedlichste Arten Karte und Daten, egal ob Raster oder Vektor, vom Benutzer angepasst und
  modifiziert werden können.%
}%
{%
  Hörsaal 3 (K0506/ Audimax II)%
}%



%%%%%%%%%%%%%%%%%%%%%%%%%%%%%%%%%%%%%%%%%%%

% time: Friday 11:10
% URL: https://pretalx.com/fossgis2023/talk/fossgis2024-39036-und-immer-wieder-lizenz-in-kompatibilitten/

%

\noindent\abstractOther{%
  Falk Zscheile%
}{%
  Und immer wieder Lizenz(in)kompatibilitäten%
}{%
}{%
  Der Vortrag beschäftigt sich mit der  Frage von Lizenz(in)kompatibilitäten der Open Database
  License zur Lizenz Creative Commons Namensnennung (CC-BY 4.0) und zur  Datenlizenz Deutschland
  Namensnennung 2.0 und legt noch einmal dar, worauf im Vorfeld einer Nutzung entsprechend
  lizenzierter Datensätze im OpenStreetMap Projekt zu achten ist.%
}%
{%
  Hörsaal 4 (A.013)%
}%



%%%%%%%%%%%%%%%%%%%%%%%%%%%%%%%%%%%%%%%%%%%

% time: Friday 11:45
% URL: https://pretalx.com/fossgis2023/talk/fossgis2024-37711-das-zusammenspiel-von-wikidata-wikipedia-und-openstreetmap/

%
\newTimeslot{11:45}
\noindent\abstractOther{%
  Christopher Lorenz%
}{%
  Das Zusammenspiel von Wikidata, Wikipedia und OpenStreetMap%
}{%
}{%
  Eine Vorstellung, welche Tags in OpenStreetMap vorhanden sind, die eine Verbindung zu Wikidata und
  Wikipedia herstellt. Zudem werden Tools und Beispiele gezeigt, wo diese Daten Anwendung finden.%
}%
{%
  Hörsaal 1 (Audimax 1)%
}%



%%%%%%%%%%%%%%%%%%%%%%%%%%%%%%%%%%%%%%%%%%%

% time: Friday 11:45
% URL: https://pretalx.com/fossgis2023/talk/fossgis2024-39025-sar-simulation-mit-raysar-perspektiven-fr-die-katastrophenhilfe/

%

\noindent\abstractOther{%
  Hannes Neuschmidt%
}{%
  SAR Simulation mit RaySAR~-- Perspektiven für die Katastrophenhilfe%
}{%
}{%
  RaySAR ist ein open source Simulationsprogramm für Synthetic Aperture Radar (SAR),
  eine hochauflösende Form von bildgebendem Radar.
  SAR Bilder sind von großem Wert in der Katastrophenhilfe um direkt nach
  einem Katastrophenfall eine Einschätzung des entstandenen Schadens zu gewinnen -
  sie können auch nachts und durch eine Wolkendecke aufgenommen werden.
  In diesem Vortrag wird RaySAR vorgestellt, sowie Ansätze, wie RaySAR bei der
  Detektion von Gebäudeschäden eingesetzt werden könnte.%
}%
{%
  Hörsaal 2 (Dietze H016)%
}%



%%%%%%%%%%%%%%%%%%%%%%%%%%%%%%%%%%%%%%%%%%%

% time: Friday 11:45
% URL: https://pretalx.com/fossgis2023/talk/fossgis2024-39004-leaflet-die-webmapping-bibliothek-die-fast-alles-kann/

%

\noindent\abstractOther{%
  Numa Gremling%
}{%
  Leaflet~-- die Webmapping-Bibliothek, die fast alles kann%
}{%
}{%
  Dieser Vortrag zeigt, dass eine Vielzahl von Plugins für Leaflet existieren, die so gut wie alle
  gewünschten Funktionen möglich machen.%
}%
{%
  Hörsaal 3 (K0506/ Audimax II)%
}%



%%%%%%%%%%%%%%%%%%%%%%%%%%%%%%%%%%%%%%%%%%%

% time: Friday 11:45
% URL: https://pretalx.com/fossgis2023/talk/fossgis2024-39047-workflow-zur-erstellung-von-trainingsdaten-fr-die-ki-gebudeerkennung/

%

\noindent\abstractOther{%
  Uwe Breitkopf, Jonas Bostelmann%
}{%
  Workflow zur Erstellung von Trainingsdaten für die KI-Gebäudeerkennung%
}{%
}{%
  Für die KI-Gebäudeerkennung in Luftbildern wird eine große Menge Trainingsdaten benötigt. In der
  Landesvermessung Niedersachsen (LGLN) wurde dafür ein Workflow unter Verwendung verschiedener Open
  Source Tools entwickelt. Damit soll der manuelle Aufwand bei der Erstellung der Trainingsdaten
  möglichst minimiert werden und ein qualitativ hochwertiger Datensatz mit hoher räumlicher
  Abdeckung entstehen, der die Variabilität der Gebäude in Niedersachsen ausreichend abbildet.%
}%
{%
  Hörsaal 4 (A.013)%
}%



%%%%%%%%%%%%%%%%%%%%%%%%%%%%%%%%%%%%%%%%%%%

% time: Friday 12:20
% URL: https://pretalx.com/fossgis2023/talk/fossgis2024-38966-jenseits-des-ndvi-hyperspektrale-fernerkundung-in-qgis-mit-der-enmap-box/

%
\newTimeslot{12:20}
\noindent\abstractOther{%
  Benjamin Jakimow%
}{%
  Jenseits des NDVI: Hyperspektrale Fernerkundung in QGIS mit der EnMAP-Box%
}{%
}{%
  Die EnMAP-Box ermöglicht eine effiziente Visualisierung und Verarbeitung von multi- und
  hyperspektralen Rasterdaten in QGIS. Sie bietet viele neue Werkzeuge zur Visualisierung von
  Rasterdaten und über 150 Algorithmen, mit denen sich umfangreiche Analyse-Workflows, etwa zur
  Abschätzung biophysikalischer Variablen, erstellen lassen. Wir stellen die EnMAP-Box vor, zeigen
  ihre neuesten Features und geben einen Ausblick auf die weiteren Entwicklungsschritte.%
}%
{%
  Hörsaal 2 (Dietze H016)%
}%



%%%%%%%%%%%%%%%%%%%%%%%%%%%%%%%%%%%%%%%%%%%

% time: Friday 12:20
% URL: https://pretalx.com/fossgis2023/talk/fossgis2024-38768-geomapfish-neues-aus-dem-vielseitigem-open-source-webgis/

%

\noindent\abstractOther{%
  Wolfgang Kaltz, Julian Hafner%
}{%
  GeoMapFish: Neues aus dem vielseitigem Open-Source-WebGIS%
}{%
}{%
  In diesem Vortrag möchten wir die neuen Entwicklungen der Open Source WebGIS-Plattform GeoMapFish
  vorstellen. Zusätzlich geben wir spannende Einblicke in den operativen Betrieb von WebGIS in
  verschiedenen Kubernetes-basierten Infrastrukturen.%
}%
{%
  Hörsaal 3 (K0506/ Audimax II)%
}%



%%%%%%%%%%%%%%%%%%%%%%%%%%%%%%%%%%%%%%%%%%%

% time: Friday 14:15
% URL: https://pretalx.com/fossgis2023/talk/fossgis2024-38841-osm-beratungsstelle-beim-fossgis-e-v-/

%
\newTimeslot{14:15}
\noindent\abstractOther{%
  Jochen Topf%
}{%
  OSM-Beratungsstelle beim FOSSGIS e.V.%
}{%
}{%
  Der FOSSGIS e.V. ist als Local Chapter der OpenStreetMap Foundation offizieller Ansprechpartner zu
  OSM-Fragen in Deutschland. Im November 2023 hat der FOSSGIS e.V. eine bezahlte Stelle für einen
  OSM-Berater eingerichtet, über deren Arbeit in diesem Vortrag berichtet werden soll.%
}%
{%
  Hörsaal 1 (Audimax 1)%
}%



%%%%%%%%%%%%%%%%%%%%%%%%%%%%%%%%%%%%%%%%%%%

% time: Friday 14:45
% URL: https://pretalx.com/fossgis2023/talk/fossgis2024-38447-fossgis-jeopardy/

%
\newTimeslot{14:45}
\noindent\abstractOther{%
  Johannes Kröger%
}{%
  FOSSGIS-Jeopardy%
}{%
}{%
  Das FOSSGIS-Jeopardy bietet wieder spannende Fragen zu (mehr oder weniger) wissenswerten Fakten
  und vor allem viel Spaß für jung und alt, alt und neu.%
}%
{%
  Hörsaal 1 (Audimax 1)%
}%



%%%%%%%%%%%%%%%%%%%%%%%%%%%%%%%%%%%%%%%%%%%

% time: Friday 15:45
% URL: https://pretalx.com/fossgis2023/talk/fossgis2024-39678-abschlussveranstaltung/

%
\newTimeslot{15:45}
\noindent\abstractOther{%
  FOSSGIS e.V.%
}{%
  Abschlussveranstaltung%
}{%
}{%
  Drei spannende Konferenztage gehen zu Ende. Ein gemeinsamer Abschluss soll erfolgen mit Rückblick
  auf die Konferenz und das Erlebte. Natürlich auch mit einem Ausblick auf kommende Veranstaltungen
  und die Konfernz im Jahr 2025.%
}%
{%
  Hörsaal 1 (Audimax 1)%
}%



%%%%%%%%%%%%%%%%%%%%%%%%%%%%%%%%%%%%%%%%%%%

% time: Friday 16:30
% URL: https://pretalx.com/fossgis2023/talk/fossgis2024-39677-sektempfang/

%
\newTimeslot{16:30}
\noindent\abstractOther{%
  FOSSGIS e.V.%
}{%
  Sektempfang%
}{%
}{%
  Der FOSSGIS e.V. lädt alle Mitglieder des FOSSGIS-Vereins, Freunde und Interessierte herzlich zum
  Sektempfang zum Ausklang der FOSSGIS 2024 am FOSSGIS-Vereins-Stand ein.%
}%
{%
  Hörsaal 1 (Audimax 1)%
}%



%%%%%%%%%%%%%%%%%%%%%%%%%%%%%%%%%%%%%%%%%%%
