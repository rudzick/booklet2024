
% time: Thursday 09:00
% URL: https://pretalx.com/fossgis2023/talk/fossgis2024-38484-masterportal-3-0-entdecke-die-magie-des-neuen-major-release-/

%
\newTimeslot{09:00}
\noindent\abstractHSeins{%
  Dirk Rohrmoser%
}{%
  Masterportal 3.0~-- Entdecke die Magie des neuen Major Release!%
}{%
}{%
  Der WebGIS Client Masterportal wurde umfassend modernisiert und ist jetzt in der neuen Major
  Version 3.0 verfügbar. Im Rahmen des Vortrags werden die Highlights der neuen Produkt-Generation
  vorgestellt.%
}%


%%%%%%%%%%%%%%%%%%%%%%%%%%%%%%%%%%%%%%%%%%%

% time: Thursday 09:00
% URL: https://pretalx.com/fossgis2023/talk/fossgis2024-38972-opendrive-hd-karten-mittels-gdal-ins-gis-bringen/

%

\noindent\abstractHSzwei{%
  Michael Scholz%
}{%
  OpenDRIVE-HD-Karten mittels GDAL ins GIS bringen%
}{%
}{%
  Ein neuer quelloffener GDAL-Treiber ermöglicht die Konvertierung detaillierter HD-Kartendaten vom
  komplexen OpenDRIVE-Format der Automobilindustrie in gängige Geodatenaustauschformate wie
  GeoPackage, Shapefile, GeoJSON, KML und räumliche Datenbanken. Dies macht OpenDRIVE endlich in
  herkömmlichen GIS-Werkzeugen nutzbar!%
}%


%%%%%%%%%%%%%%%%%%%%%%%%%%%%%%%%%%%%%%%%%%%

% time: Thursday 09:00
% URL: https://pretalx.com/fossgis2023/talk/fossgis2024-38887-ermittlung-von-solarpotentialflchen-auf-gebuden/

%

\noindent\abstractHSdrei{%
  Sarah Schütz, Elena Zentgraf%
}{%
  Ermittlung von Solarpotentialflächen auf Gebäuden%
}{%
}{%
  Im Vortrag wird gezeigt, wie eine Analyse der aktuellen und potentiellen Situation der
  Solarenergienutzung eine mögliche Grundlage für den Ausbau erneuerbaren Energien sein kann. Bei
  dem vorgestellten Vorgehen werden dabei nur Open Data und Open Source Tools verwendet.%
}%


%%%%%%%%%%%%%%%%%%%%%%%%%%%%%%%%%%%%%%%%%%%

% time: Thursday 09:00
% URL: https://pretalx.com/fossgis2023/talk/fossgis2024-38920-analyse-und-visualisierung-von-polizeimeldungen-eine-fallstudie-fr-frankfurt-a-m-/

%

\noindent\abstractHSvier{%
  Johannes Frank%
}{%
  Analyse und Visualisierung von Polizeimeldungen: Eine Fallstudie für Frankfurt a. M.%
}{%
}{%
  Polizeimeldungen von Polizeibehörden informieren unter anderem über Straftaten ohne genaue
  Ortsangaben. Ein Prozessablauf kombiniert Open Source Software und OpenStreetMap-Daten, um die
  Meldungen räumlich darzustellen. Offene Standards ermöglichen die einfache Nutzung der Geodaten.
  Der Vortrag zeigt Ergebnisse für Frankfurt am Main und diskutiert Herausforderungen sowie mögliche
  Verbesserungen.%
}%


%%%%%%%%%%%%%%%%%%%%%%%%%%%%%%%%%%%%%%%%%%%

% time: Thursday 09:00
% URL: https://pretalx.com/fossgis2023/talk/fossgis2024-38905-qgis-in-der-ffentlichen-verwaltung/

%

\noindent\abstractAnwBoFeins{%
  David Arndt%
}{%
  QGIS in der Öffentlichen Verwaltung%
}{%
}{%
  QGIS erfreut sich einer immer größeren Beliebtheit in der Öffentlichen Verwaltung. Die Vernetzung
  der Kommunen ist hier ein wichtiger Bestandteil für die erfolgreiche Einführung von QGIS. Die
  Session soll dazu dienen Erfahrungen auszutauschen. Geeignet für Neueinsteiger und Profis.%
}%


%%%%%%%%%%%%%%%%%%%%%%%%%%%%%%%%%%%%%%%%%%%

% time: Thursday 09:00
% URL: https://pretalx.com/fossgis2023/talk/fossgis2024-38842-bof-geonode-de/

%

\noindent\abstractAnwBoFzwei{%
  Florian Hoedt, Henning Bredel%
}{%
  BoF GeoNode-DE%
}{%
}{%
  Die Gruppe GeoNode-DE möchte gemeinsam Fragen zur Weiterentwicklung diskutieren und das vorhandene
  Netzwerk zu stärken.%
}%


%%%%%%%%%%%%%%%%%%%%%%%%%%%%%%%%%%%%%%%%%%%

% time: Thursday 09:35
% URL: https://pretalx.com/fossgis2023/talk/fossgis2024-38984-das-beispiel-masterportal-ein-os-erfolgsmodell-fr-die-ffentliche-verwaltung-/

%
\newTimeslot{09:35}
\noindent\abstractHSeins{%
  Nicholas Schliffke%
}{%
  Das Beispiel Masterportal~-- ein OS Erfolgsmodell für die öffentliche Verwaltung?%
}{%
}{%
  Das Masterportal hat sich in den letzten Jahren innerhalb der deutschen Verwaltung als Geoviewer
  bzw. WebGIS etabliert. In diesem Vortrag werden die erfolgreichen Organisationsstrukturen und
  Prozesse hinter dem Masterportal vorgestellt und diskutiert, ob dieses Open-Source Modell als
  Blaupause weiterer Projekte der öffentlichen Verwaltung oder darüber hinaus dienen könnte.%
}%


%%%%%%%%%%%%%%%%%%%%%%%%%%%%%%%%%%%%%%%%%%%

% time: Thursday 09:35
% URL: https://pretalx.com/fossgis2023/talk/fossgis2024-39018-skalierbare-geodatenverarbeitung-in-der-cloud-mit-argo-workflows/

%

\noindent\abstractHSzwei{%
  Frederik Diehl, Holger Bach%
}{%
  Skalierbare Geodatenverarbeitung in der Cloud mit Argo Workflows%
}{%
}{%
  Auch von amtlichen Geodaten wird heutzutage eine hohe Aktualität erwartet. Aus diesem Grund setzt
  die Landesvermessung Niedersachsen (LGLN) bei der Verarbeitung ihrer Geodaten vermehrt auf
  vollständig automatisierte Prozesse. Um trotz wachsender Datenmengen in Zukunft möglichst schnell
  aktuelle Geodaten für ganz Niedersachsen liefern zu können, muss die Verarbeitung in skalierbaren
  Cloud-Umgebungen erfolgen. Zur Orchestrierung setzen wir Kubernetes und die Workflow-Engine Argo
  Workflows ein.%
}%


%%%%%%%%%%%%%%%%%%%%%%%%%%%%%%%%%%%%%%%%%%%

% time: Thursday 09:35
% URL: https://pretalx.com/fossgis2023/talk/fossgis2024-38495-virtuelle-erreichbarkeitsanalysen-in-kommonitor-zur-sozialrumlichen-bedarfsplanung/

%

\noindent\abstractHSdrei{%
  Sebastian Drost, Isabell Rohling, Christian Danowski-Buhren%
}{%
  Virtuelle Erreichbarkeitsanalysen in KomMonitor zur sozialräumlichen Bedarfsplanung%
}{%
}{%
  Die Open-Source Software KomMonitor bietet Möglichkeiten zur Durchführung von
  Erreichbarkeitsanalysen auf Basis eines OSM-Netzwerks. Das integrierte Tool zur Berechnung von
  Erreichbarkeitsisochronen wurde nun um eine virtuelle Szenarienfunktion ergänzt. Diese unterstützt
  kommunale Fach- und Sozialplaner:innen bei der quantifizierten Analyse der Nahversorgung durch
  Infrastruktureinrichtungen in kleinräumigen Gebieten.%
}%


%%%%%%%%%%%%%%%%%%%%%%%%%%%%%%%%%%%%%%%%%%%

% time: Thursday 09:35
% URL: https://pretalx.com/fossgis2023/talk/fossgis2024-38929-keine-angst-vor-sperrigen-ausdrcken-im-qgis-/

%

\noindent\abstractHSvier{%
  Claas Leiner%
}{%
  Keine Angst vor sperrigen Ausdrücken im QGIS!%
}{%
}{%
  Keine Angst vor Ausdrücken im QGIS!
  Live geht es um:
  -Case und if für Bedigungen.
  -Mit coalesce(), concat() etc. nie mehr über NULL-Werte stolpern.
  -Joins mit attribute() und get\_feature() simulieren.
  -1:N-Beziehungen und räumliche Abfragen mit aggregate() und overlay() umsetzen.
  -Was hat es mit diesen Arrays auf sich?.
  -Mit with\_variable()  Ausdrücke lesbar gestalten.
  -Geometriefunktionen in Berechnungen integrieren.
  -Wie helfen diese merkwürdigen regulären Ausdrücke bei regexp\_..()%
}%


%%%%%%%%%%%%%%%%%%%%%%%%%%%%%%%%%%%%%%%%%%%

% time: Thursday 10:10
% URL: https://pretalx.com/fossgis2023/talk/fossgis2024-38784-das-masterportal-als-bestandteil-der-open-smartcity-solingen/

%
\newTimeslot{10:10}
\noindent\abstractHSeins{%
  Markus Stein, Shakti Gahlaut%
}{%
  Das Masterportal als Bestandteil der Open SmartCity Solingen%
}{%
}{%
  Die Klingenstadt Solingen ist im Rahmen einer Förderung des Bundesinnenministeriums seit Januar
  2020 eines der SmartCity Modellprojekte. Die Förderung endet im Herbst 2024.
  Im Zuge des Modellprojektes wird das GEOportal der Klingenstadt auf das Masterportal umgestellt.
  Vorgestellt werden sollen insbesondere das Deployment des Masterportals über eine CI/CD-Pipeline
  über einen eigenen gitlab-Server und die Einbindung der Sensordaten aus dem FROST-Server, der
  ebenfalls ein Teil des Projekts ist.%
}%


%%%%%%%%%%%%%%%%%%%%%%%%%%%%%%%%%%%%%%%%%%%

% time: Thursday 10:10
% URL: https://pretalx.com/fossgis2023/talk/fossgis2024-38785-verarbeitung-hochaufgelster-umweltdaten-auf-basis-von-ogc-api-processes/

%

\noindent\abstractHSzwei{%
  Dr. Svenja Dobbert%
}{%
  Verarbeitung hochaufgelöster Umweltdaten auf Basis von OGC API Processes%
}{%
}{%
  Das Forschungsprojekt KLIPS hat zum Ziel hochaufgelöste Umweltdaten in nutzbarer Form zur
  Verfügung zu stellen, um eine inhaltliche Interpretation für Gemeinden und Städte zu ermöglichen.
  Hier kommt eine umfangreiche Geodateninfrastruktur zum Einsatz, welche Temperaturdaten empfängt,
  prozessiert, analysiert, und anschließend in Form von browserbasierten Demonstratoren darstellt.
  Der Vortrag stellt die Verarbeitung solcher Daten u.a. mit Hilfe von OGC API Processes und
  pygeoapi vor.%
}%


%%%%%%%%%%%%%%%%%%%%%%%%%%%%%%%%%%%%%%%%%%%

% time: Thursday 10:10
% URL: https://pretalx.com/fossgis2023/talk/fossgis2024-39020-rumliche-fragmentierung-im-v-angebot-sichtbar-machen-dank-offenen-fahrplandaten/

%

\noindent\abstractHSdrei{%
  Theodor Rieche%
}{%
  Räumliche Fragmentierung im ÖV-Angebot sichtbar machen~-- dank offenen Fahrplandaten%
}{%
}{%
  Basierend auf offenen Fahrplandaten im GTFS-Format wird der angebotene Fahrplan des öffentlichen
  Verkehrs hinsichtlich der Vernetzung über Zuständigkeitsgrenzen (wie Landkreise, Verkehrsverbünde,
  Bundesländer) hinweg analysiert. Zentrale Frage dabei ist, ob solche Grenzen als räumliche
  Barriere im angebotenen Fahrplan wirken. So soll ein Beitrag zur aktuellen Debatte um die
  Verkehrswende geleistet werden.%
}%


%%%%%%%%%%%%%%%%%%%%%%%%%%%%%%%%%%%%%%%%%%%

% time: Thursday 11:10
% URL: https://pretalx.com/fossgis2023/talk/fossgis2024-38537-basemap-de-aktuelles-und-ausblick/

%
\newSmallTimeslot{11:10}
\noindent\abstractHSeins{%
  Arnulf B. Christl%
}{%
  basemap.de Aktuelles und Ausblick%
}{%
}{%
  Im Vortrag wird der aktuelle Stand des Projekts basemap.de der amtlichen deutschen Vermessung
  (AdV) vorgestellt. Es handelt sich um mehrere kostenfrei nutzbare Dienste und zunehmend auch der
  entsprechenden Quelldaten, die in regelmäßigen, kurzen Abständen aktualisiert zur Verfügung
  stehen.%
}%


%%%%%%%%%%%%%%%%%%%%%%%%%%%%%%%%%%%%%%%%%%%

% time: Thursday 11:10
% URL: https://pretalx.com/fossgis2023/talk/fossgis2024-38717-qgis-server-einsatz-im-unternehmen/

%
\newpage
\noindent\abstractHSzwei{%
  Jakob Miksch%
}{%
  QGIS Server~-- Einsatz im Unternehmen%
}{%
}{%
  Um Geodaten über das Web verfügbar zu machen, nutzen wir bei siticom unter anderem QGIS Server.
  Der Vortrag beleuchtet das verwendete Setup um eine Vielzahl von verschiedenen Projekten zu
  veröffentlichen und erläutert Vor- und Nachteile im Vergleich zu anderen gängigen Lösungen wie
  GeoServer.%
}%


%%%%%%%%%%%%%%%%%%%%%%%%%%%%%%%%%%%%%%%%%%%

% time: Thursday 11:10
% URL: https://pretalx.com/fossgis2023/talk/fossgis2024-39051-gemeinsam-gebudeinformationen-erfassen-im-citizen-science-projekt-colouring-dresden/

%

\noindent\abstractHSdrei{%
  Theodor Rieche%
}{%
  Gemeinsam Gebäudeinformationen erfassen im Citizen-Science-Projekt Colouring Dresden%
}{%
}{%
  Um fehlende Daten zu Gebäuden wie Alter, Baumaterial oder Nutzung insbesondere für Wissenschaft
  und Planung erfassen und als offene Daten bereitstellen zu können, wurde in einem
  Citizen-Science-Projekt die Plattform "`Colouring Dresden"' erprobt und weiterentwickelt. In einer
  interaktiven Karte können derzeit 40 Gebäudemerkmale in sieben Kategorien gemeinschaftlich erfasst
  werden. Der Vortrag geht auf die Erfahrungen aus dem Projekt ein und stellt die verwendeten
  technischen Komponenten vor.%
}%


%%%%%%%%%%%%%%%%%%%%%%%%%%%%%%%%%%%%%%%%%%%

% time: Thursday 11:10
% URL: https://pretalx.com/fossgis2023/talk/fossgis2024-39001-projekt-geodigitalisierungskomponente-gdik-/

%
\newpage
\noindent\abstractHSvier{%
  Kai Culemann, Jannik Günther%
}{%
  Projekt Geodigitalisierungskomponente (GDIK)%
}{%
}{%
  50 \% aller Daten haben einen Raumbezug. Daraus folgt, dass mindestens 50 \% aller Formulare einen
  Raumbezug haben, aber im HTML Standard existiert kein Formular-Element für die Eingabe von
  Geodaten.
  Im Rahmen des OZG Projektes Geodigitalisierungskomponente haben wir uns dieser Thematik
  angenommen, in Online-Formularen Geometrien angeb- und auswählbar zu machen.
  Das Projekt GDIK setzt auf Web-Components und der MasterportalAPI als technischen Unterbau.%
}%


%%%%%%%%%%%%%%%%%%%%%%%%%%%%%%%%%%%%%%%%%%%

% time: Thursday 11:10
% URL: https://pretalx.com/fossgis2023/talk/fossgis2024-38022-ask-me-anything-openeo-api-prozesse-und-kosystem/

%

\noindent\abstractExp{%
  Matthias Mohr%
}{%
  Ask me anything: openEO~-- API, Prozesse und Ökosystem%
}{%
}{%
  openEO entwickelt eine offene API, um diverse Klienten auf einfache und einheitliche Weise mit
  großen EO-Cloud-Diensten zu verbinden. Geoprozessierung soll so einfacher, reproduzierbar und
  zugänglich gemacht werden. Die Spezifkation und das Ökosystem haben mittlerweile einen großen
  Umfang erreicht. Haben Sie Fragen? Gerne versuchen wir diese zu beantworten...%
}%


%%%%%%%%%%%%%%%%%%%%%%%%%%%%%%%%%%%%%%%%%%%

% time: Thursday 11:10
% URL: https://pretalx.com/fossgis2023/talk/fossgis2024-38922-neues-von-der-gbd-websuite/

%
\newpage
\noindent\abstractAnwBoFeins{%
  Otto Dassau%
}{%
  Neues von der GBD WebSuite%
}{%
}{%
  Im Sommer diesen Jahres ist die neue Version 8.0 der GBD Web\-Suite veröffentlicht worden. Es
  handelt sich hierbei um eine umfangreiche Überarbeitung. Die Software-Architektur wurde
  grundlegend umgebaut, der Quellcode und das Datenmodell überarbeitet und Funktionalität erweitert
  und optimiert. Diese umfangreichen Neuerungen möchten wir gerne vorstellen.%
}%


%%%%%%%%%%%%%%%%%%%%%%%%%%%%%%%%%%%%%%%%%%%

% time: Thursday 11:30
% URL: https://pretalx.com/fossgis2023/talk/fossgis2024-38923-gbd-websuite-anwendertreffen/

%
\newSmallTimeslot{11:30}
\noindent\abstractAnwBoFeins{%
  Otto Dassau%
}{%
  GBD WebSuite Anwendertreffen%
}{%
}{%
  Die GBD WebSuite ist eine Open Source WebGIS Plattform zur Geodatenverarbeitung
  (https://gbd-websuite.de). Sie wird seit 2017 entwickelt und deutschlandweit von Kommunen und
  privaten Unternehmen genutzt. Wir möchten die FOSSGIS für den Austausch zwischen Anwendern nutzen
  und Interessierten das Open Source Projekt vorstellen.%
}%


%%%%%%%%%%%%%%%%%%%%%%%%%%%%%%%%%%%%%%%%%%%

% time: Thursday 11:45
% URL: https://pretalx.com/fossgis2023/talk/fossgis2024-38817-open-data-des-bkg/

%
\newSmallTimeslot{11:45}
\noindent\abstractHSeins{%
  Joachim Eisenberg%
}{%
  Open Data des BKG%
}{%
}{%
  Das Bundesamt für Kartographie und Geodäsie (BKG) ist der Geodatendienstleister des Bundes. Neben
  der Entwicklung von Geoprodukten für Bundesbehörden ist es jedoch auch bestrebt, Geodaten als Open
  Data allen interessierten Nutzern zur Verfügung zu stellen.%
}%


%%%%%%%%%%%%%%%%%%%%%%%%%%%%%%%%%%%%%%%%%%%

% time: Thursday 11:45
% URL: https://pretalx.com/fossgis2023/talk/fossgis2024-38715-qgis-server-plugins/

%

\noindent\abstractHSzwei{%
  Marco Hugentobler%
}{%
  QGIS Server Plugins%
}{%
}{%
  Die Erweiterung von QGIS mit Python-Plugins bietet eine Vielzahl von Möglichkeiten, die Software
  auf den eigenen Anwendungsfall anzupassen und ist dementsprechend populär. Im Gegensatz zu den
  Plugins für QGIS Desktop sind die Plugins für QGIS Server weniger bekannt. Dieser Vortrag soll das
  ändern. Zuerst wird auf die Technik eingegangen und erläutert, wie ein Plugin mit dem Server
  interagiert. Dann werden einige Anwendungsfälle und Plugins vorgestellt, die für den Serverbetrieb
  nützlich sind.%
}%


%%%%%%%%%%%%%%%%%%%%%%%%%%%%%%%%%%%%%%%%%%%

% time: Thursday 11:45
% URL: https://pretalx.com/fossgis2023/talk/fossgis2024-38014-ableitung-von-korrekten-osm-rumen-wnden-und-tren-aus-ifc-gebudemodellen/

%

\noindent\abstractHSdrei{%
  Helga Tauscher%
}{%
  Ableitung von korrekten OSM-Räumen, -Wänden und -Türen aus IFC-Gebäudemodellen%
}{%
}{%
  In diesem Paper diskutieren wir die Ableitung von Inneraumdaten im Format OSM-SIT aus digitalen
  Gebäubemodellen im Format IFC. Dabei zeigen wir insbesondere Probleme und Lösungsansätze
  hinsichtlich der Topologie und Geometrie von Räumen, Raumgrenzen und -verbindungen zwischen Räumen
  innerhalb eines Geschosses.%
}%


%%%%%%%%%%%%%%%%%%%%%%%%%%%%%%%%%%%%%%%%%%%

% time: Thursday 11:45
% URL: https://pretalx.com/fossgis2023/talk/fossgis2024-39056-comaps-die-planungssuite-fr-das-masterportal/

%
\newpage
\noindent\abstractHSvier{%
  Daniel Schulz%
}{%
  comaps~-- die Planungssuite für das Masterportal%
}{%
}{%
  comaps ist ein umfangreiches Addon-Paket für das OpenSource Masterportal, das Visualisierung und
  Analyse städtischer Statistik- und Strukturdaten vereinfacht, um transparente und fundierte
  Entscheidungen in der Planung zu unterstützen. Es verknüpft Daten verschiedener Quellen wie
  Fachbehörden und Unternehmen und integriert räumliche und zeitliche Analysen, sowie
  Szenarioplanung. Mit interaktiven Karten und benutzerfreundlichen Tools unterstützt comaps eine
  vernetzte und datengesteuerte Planung.%
}%


%%%%%%%%%%%%%%%%%%%%%%%%%%%%%%%%%%%%%%%%%%%

% time: Thursday 12:20
% URL: https://pretalx.com/fossgis2023/talk/fossgis2024-38993-offene-verwaltungsdaten-in-europa-was-deutschland-von-anderen-lndern-lernen-kann/

%
\newSmallTimeslot{12:20}
\noindent\abstractHSeins{%
  Marina Happ%
}{%
  Offene Verwaltungsdaten in Europa: Was Deutschland von anderen Ländern lernen kann%
}{%
}{%
  Der Beitrag des Wissenschaftlichen Instituts für Infrastruktur und Kommunikationsdienste (WIK)
  untersucht das Angebot von offenen Verwaltungsdaten in Europa und welche Rolle zentrale
  Open-Data-Institutionen dabei spielen. Die Ergebnisse zeigen, dass andere Länder schneller als
  Deutschland voranschreiten. Sie stellen u. a. weniger Bedingungen an die Datennutzung und bieten
  deutlich mehr Geo-, Tabellen- und Textformate anstatt Bilddaten an.%
}%


%%%%%%%%%%%%%%%%%%%%%%%%%%%%%%%%%%%%%%%%%%%

% time: Thursday 12:20
% URL: https://pretalx.com/fossgis2023/talk/fossgis2024-38755-host-your-own-qgis-plugin-repository/

%

\noindent\abstractHSzwei{%
  Frida Kessler%
}{%
  Host your own QGIS Plugin\linebreak Repository%
}{%
}{%
  In diesem Vortrag zeige ich euch wie das QGIS Plugin Repository funktioniert und wie ihr euer
  eigenes. QGIS Plugin Repository hosten könnt.%
}%


%%%%%%%%%%%%%%%%%%%%%%%%%%%%%%%%%%%%%%%%%%%

% time: Thursday 12:20
% URL: https://pretalx.com/fossgis2023/talk/fossgis2024-38714-barrierefreie-indoor-navigation-auf-basis-von-osm-daten/

%

\noindent\abstractHSdrei{%
  René Apitzsch, Robin Thomas%
}{%
  Barrierefreie Indoor-Navigation auf Basis von OSM-Daten%
}{%
}{%
  In diesem Vortrag wird auf die Entwicklung und die damit verbundenen Herausforderungen einer
  barrierefreien Indoor-Navigations-App eingegangen. Grundlage für diese App sind OSM-Daten zu
  Barrieren und Eigenschaften von Haltestellen im Öffentlichen Personenverkehr. Dieser Vortrag ist
  eine Fortführung zu unserem Beitrag von der FOSSGIS 2022 [1], in dem präsentiert wurde, wie im
  Projekt OPENER next mit Hilfe der App OpenStop jene Daten er-fasst werden.%
}%


%%%%%%%%%%%%%%%%%%%%%%%%%%%%%%%%%%%%%%%%%%%

% time: Thursday 14:15
% URL: https://pretalx.com/fossgis2023/talk/fossgis2024-39041-ki-gebudeerkennung-deep-learning-modelle-zur-aktualisierung-der-alkis-gebude/

%
\newTimeslot{14:15}
\noindent\abstractHSeins{%
  Jonas Bostelmann, Birger Giesen, Valentina Schmidt%
}{%
  KI-Gebäudeerkennung~-- Deep-Learning-Modelle zur Aktualisierung der ALKIS-Gebäude%
}{%
}{%
  Beim Einsatz "`Künstlicher Intelligenz” zur Erkennung von Gebäuden in Luftbildern setzt die
  Landesvermessung Niedersachsen (LGLN) auf Open Source Software und selbst trainierte
  Deep-Learning-Modelle. Ein eigenes DecSecOps-Team entwickelt und betreibt seit über 4 Jahren eine
  SaaS-Anwendung zur Unterstützung der Katasterämter. Diese "`KI-Gebäudeerkennung” hilft beim
  Aktualisieren der ALKIS-Daten. Kann sie auch beim Aktualisieren der OSM-Gebäude helfen?%
}%


%%%%%%%%%%%%%%%%%%%%%%%%%%%%%%%%%%%%%%%%%%%

% time: Thursday 14:15
% URL: https://pretalx.com/fossgis2023/talk/fossgis2024-39030-qfield-3-feldarbeit-neu-definiert/

%

\noindent\abstractHSzwei{%
  Marco Bernasocchi%
}{%
  QField 3~-- Feldarbeit neu definiert%
}{%
}{%
  Die wichtigsten zwischen März 2023 und 2024 entwickelten Features für die Feldapplikation QField
  werden vorgestellt. Dazu gehören die auf QT6 basierend 3.0 Release mit der Möglichkeit NFC
  einzulesen, Point Cloud Profile, erweiterte Suche, erweiterte Tracking- und Snapping- Kontrolle
  und vieles mehr%
}%


%%%%%%%%%%%%%%%%%%%%%%%%%%%%%%%%%%%%%%%%%%%

% time: Thursday 14:15
% URL: https://pretalx.com/fossgis2023/talk/fossgis2024-38837-mit-openrouteservice-zu-routingplus-einblicke-in-einen-globalen-routing-cluster/

%
\newpage
\noindent\abstractHSdrei{%
  Florian Micklich, Julian Psotta%
}{%
  Mit openrouteservice zu RoutingPlus~-- Einblicke in einen globalen Routing-Cluster%
}{%
}{%
  Mit openrouteservice zu RoutingPlus. Die langjährige Zusammenarbeit des Heidelberg Institute for
  Geoinformation Technology (HeiGIT) mit dem Bundesamt für Kartographie und Geodäsie (BKG) zum Thema
  Routing hat bereits viele technische Neuerungen hervorgebracht. Mit fortschreitender Zeit stehen
  nun Herausforderungen bei Modernisierung und Anpassungen an, um die Anforderungen eines modernen
  Softwarebetriebs zu erfüllen. In unserem Vortrag möchten wir euch auf unseren Weg dorthin
  mitnehmen.%
}%


%%%%%%%%%%%%%%%%%%%%%%%%%%%%%%%%%%%%%%%%%%%

% time: Thursday 14:15
% URL: https://pretalx.com/fossgis2023/talk/fossgis2024-39013-wkt2-proj-benutzerdefinierte-koordinatensysteme-am-beispiel-von-pix4dcatch/

%

\noindent\abstractHSvier{%
  Alexander Nehrbaß, Javier Jimenez Shaw%
}{%
  WKT2+PROJ: Benutzerdefinierte Koordinatensysteme am Beispiel von PIX4Dcatch%
}{%
}{%
  Mithilfe des offenen WKT2 (well-known text) Standards und der FOSS Bibliothek PROJ, zeigen wir wie
  benutzerdefinierte Koordinatenreferenzsysteme (CRS) definiert und Koordinaten zu und von anderen
  Referenzsystemen transformiert werden können. Als Anwendungsfall dient das Kalibrieren eines
  lokalen Koordinatenreferenzsystems am Beispiel von PIX4Dcatch.%
}%


%%%%%%%%%%%%%%%%%%%%%%%%%%%%%%%%%%%%%%%%%%%

% time: Thursday 14:15
% URL: https://pretalx.com/fossgis2023/talk/fossgis2024-38725-was-machen-kartographen-heutezutage-/

%

\noindent\abstractAnwBoFeins{%
  Mathias Gröbe, Jochen Schiewe%
}{%
  Was machen Kartographen heutezutage?%
}{%
}{%
  Keine Geodaten ohne Visualisierung~-- die Kartographie macht erst sichtbar, was alles in den Daten
  steckt. Die Deutsche Gesellschaft für Kartographie möchte in Ihrem etwas anderen Anwendertreffen
  einen Einblick in Ihre Arbeit geben und vorstellen, wo heute Kartographen arbeiten und was Sie
  alles machen.%
}%


%%%%%%%%%%%%%%%%%%%%%%%%%%%%%%%%%%%%%%%%%%%

% time: Thursday 14:15
% URL: https://pretalx.com/fossgis2023/talk/fossgis2024-38730-osgeo-deegree-anwendertreffen/

%

\noindent\abstractAnwBoFzwei{%
  Torsten Friebe%
}{%
  OSGeo deegree~-- Anwendertreffen%
}{%
}{%
  Zum Anwendertreffen sind Anwender:innen und Entwickler:innen herzlich eingeladen, die
  Netzwerkdienste wie WMS und WFS mit dem [OSGeo-Projekt deegree](https://www.deegree.org/) bereits
  umsetzen oder dieses für die Zukunft planen.%
}%


%%%%%%%%%%%%%%%%%%%%%%%%%%%%%%%%%%%%%%%%%%%

% time: Thursday 14:50
% URL: https://pretalx.com/fossgis2023/talk/fossgis2024-39050-evaluierung-von-hausumringen-alkis-osm-microsoft-und-unsere-ki-im-vergleich/

%
\newTimeslot{14:50}
\noindent\abstractHSeins{%
  Lukas Sanner, Mike Engel%
}{%
  Evaluierung von Hausumringen: ALKIS, OSM, Microsoft und unsere KI im Vergleich%
}{%
}{%
  Um automatisiert Hinweise zur Aktualisierung der ALKIS-Daten zu erhalten, werden bei der
  Landesvermessung Niedersachsen (LGLN) die Ergebnisse einer "`KI-Gebäudeerkennung"' mit den amtlichen
  Hausumringen verglichen. Dazu wurde eine Reihe von Metriken entwickelt.
  Mit denselben Metriken haben wir auch andere frei verfügbare Datensätze (z.B.: OSM) ausgewertet,
  bestehende Differenzen analysiert und daraus Aussagen über Vollständigkeit, Aktualität und
  Genauigkeit der jeweiligen Hausumringe abgeleitet.%
}%


%%%%%%%%%%%%%%%%%%%%%%%%%%%%%%%%%%%%%%%%%%%

% time: Thursday 14:50
% URL: https://pretalx.com/fossgis2023/talk/fossgis2024-38564-qkan-kanalkataster-und-planungssystem-fr-qgis/

%

\noindent\abstractHSzwei{%
  Jörg Höttges%
}{%
  QKan~-- Kanalkataster und Planungssystem für QGIS%
}{%
}{%
  Das Planungssystem "`QKan"' wird seit nun bereits 7 Jahren entwickelt. Es dient der
  Datenaufbereitung für und der Ergebnisdarstellung von hydraulischen Simulationen zu kommunalen
  Entwässerungssystemen und beinhaltet zahlreiche Datenaustauschformate. Neu entwickelt werden
  aktuell die Zustandserfassung und -bewertung von Kanalnetzen sowie die 2D- und
  3D-Ergebnisdarstellung von gekoppelten Kanalnetz- und Oberflächenabflusssimulationen.%
}%


%%%%%%%%%%%%%%%%%%%%%%%%%%%%%%%%%%%%%%%%%%%

% time: Thursday 14:50
% URL: https://pretalx.com/fossgis2023/talk/fossgis2024-38897-geografische-postgresql-erweiterungen-pgrouting-und-postgis/

%

\noindent\abstractHSdrei{%
  Marion Baumgartner, Julian Hafner%
}{%
  Geografische PostgreSQL Erweiterungen: pgRouting und PostGIS%
}{%
}{%
  In diesem Vortrag werden wir einen zielführenden Einblick in pgRouting und PostGIS geben und
  anhand von Beispielen aus der Praxis einen kurzen Weg durch mögliche Einsatzgebiete von pgRouting
  finden.%
}%


%%%%%%%%%%%%%%%%%%%%%%%%%%%%%%%%%%%%%%%%%%%

% time: Thursday 14:50
% URL: https://pretalx.com/fossgis2023/talk/fossgis2024-38689-ogc-api-features-mit-mapserver/

%

\noindent\abstractHSvier{%
  Jörg Thomsen%
}{%
  OGC API: Features mit MapServer%
}{%
}{%
  Seit Version 8 unterstützt MapServer auch die OGC API: Features, weitere Standards werden sicher
  folgen. In der Demo-Session wird gezeigt, wie man MapServer 8 so konfiguriert, dass die eigenen
  Daten auch über die OGC Api bereit gestellt werden können und wie man die Landing Page individuell
  gestalten kann.%
}%


%%%%%%%%%%%%%%%%%%%%%%%%%%%%%%%%%%%%%%%%%%%

% time: Thursday 15:25
% URL: https://pretalx.com/fossgis2023/talk/fossgis2024-38468-ki-gis-eo-foss-erfahrungen-offene-fragen-rund-um-artifizielle-intelligenz/

%
\newTimeslot{15:25}
\noindent\abstractHSeins{%
  Marc Jansen, Carmen Tawalika%
}{%
  KI, GIS, EO \& FOSS: Erfahrungen \& offene Fragen rund um artifizielle Intelligenz%
}{%
}{%
  Im Vortrag wollen wir die faszinierende Welt der künstlichen Intelligenz (KI) im Kontext von
  Geoinformationssystemen (GIS) \& Earth Observation (EO) betrachten.
  Gemeinsam wollen wir (keine KI-Experten, aber reichlich Praxiserfahrung) Begriffe klären, konkrete
  Anwendungsbeispiele vorstellen und dabei auch wichtige offene Fragen zur Diskussion stellen.
  Dieser Vortrag nimmt Sie mit auf eine Reise durch die spannende Umbruchszeit, in der sich auch
  unsere Branche befindet und soll Diskurs anregen.%
}%


%%%%%%%%%%%%%%%%%%%%%%%%%%%%%%%%%%%%%%%%%%%

% time: Thursday 15:25
% URL: https://pretalx.com/fossgis2023/talk/fossgis2024-38995-zerstrte-stdte-historische-karten-des-zweiten-weltkriegs-in-qgis-analysieren/

%

\noindent\abstractHSzwei{%
  Anastasia Bauch, Klaus Stein%
}{%
  Zerstörte Städte: Historische Karten des Zweiten Weltkriegs in QGIS analysieren%
}{%
}{%
  Historische thematische Karten auf Basis der gleichen Stadtgrundkarte bilden eine
  Multi-Layer-Geodatenbank auf Papier. Wir zeigen, wie wir diese Kartensätze in QGIS als Thick/Deep
  Map erfassen und für die denkmalwissenschaftliche Forschung aufbereiten und nutzen.%
}%


%%%%%%%%%%%%%%%%%%%%%%%%%%%%%%%%%%%%%%%%%%%

% time: Thursday 15:25
% URL: https://pretalx.com/fossgis2023/talk/fossgis2024-38786-der-elefant-kann-s-auch-allein-graph-erstellung-aus-osm-in-der-postgis-datenbank-/

%

\noindent\abstractHSdrei{%
  Matthias Daues%
}{%
  Der Elefant kann's auch allein: Graph-Erstellung aus OSM in der PostGIS-Datenbank.%
}{%
}{%
  Vom osm-dump zum voll vernetzten Graphen: Mit osmium, osm2\-pgsql und einigen simplen
  Datenbank-Prozeduren gelingt die Umwandlung von rein geographischen Informationen in logische
  Datenstrukturen.
  Im GitHub findest Du alles, was Du zum selber machen brauchst.%
}%


%%%%%%%%%%%%%%%%%%%%%%%%%%%%%%%%%%%%%%%%%%%

% time: Thursday 16:45
% URL: https://pretalx.com/fossgis2023/talk/fossgis2024-38909-altkartenanalyse-fr-einen-nachhaltigen-klimaschutz-entwicklung-eines-qgis-plugins/

%
\newSmallTimeslot{16:45}
\noindent\abstractHSeins{%
  Eszter Kiss, André Hartmann, Hendrik Herold%
}{%
  Altkartenanalyse für einen nachhaltigen Klimaschutz~-- Entwicklung eines QGIS-Plugins%
}{%
}{%
  In einem neuen Gemeinschaftsprojekt zwischen IÖR und BKG wird untersucht, in welchem Maße sich die
  Landbedeckung in Deutschland vom Anfang des 19. Jahrhunderts bis heute verändert hat. Dazu werden
  Methoden erarbeitet, um Karteninhalte aus digitalisierten Altkartenbeständen automatisiert zu
  extrahieren.Die im Projekt entstehenden Ergebnisse werden als offene Geodaten, die Methoden in
  einem QGIS Plug-In bereitgestellt.%
}%


%%%%%%%%%%%%%%%%%%%%%%%%%%%%%%%%%%%%%%%%%%%

% time: Thursday 16:45
% URL: https://pretalx.com/fossgis2023/talk/fossgis2024-38328-qgis-js-qgis-im-browser-dank-webassembly/

%
\newpage
\noindent\abstractHSzwei{%
  Michael Schmuki, Andreas Neumann%
}{%
  qgis-js~-- QGIS im Browser dank WebAssembly%
}{%
}{%
  qgis-js ist eine Portierung von QGIS Core zu WebAssembly um es in modernen Browsern auszuführen.
  Dieses Setup ermöglicht die Integration von praktisch allen denkbaren Geo-Formaten und dynamische
  kartografische Darstellungen auf höchsten Niveau ganz ohne (QGIS-)Server. Im Rahmen des Vortrags
  werden die verwendeten Technologien sowie die Architektur kurz vorgestellt, um anschliessend die
  neuen Möglichkeiten und Integration anhand interaktiven Beispielen aufzuzeigen.%
}%


%%%%%%%%%%%%%%%%%%%%%%%%%%%%%%%%%%%%%%%%%%%

% time: Thursday 16:45
% URL: https://pretalx.com/fossgis2023/talk/fossgis2024-38974-user-analyse-mit-der-overpass-api-zwischen-vandalismus-verfolgung-und-stalking/

%

\noindent\abstractHSdrei{%
  Dr. Roland Olbricht%
}{%
  User-Analyse mit der Overpass API: Zwischen Vandalismus-Verfolgung und Stalking%
}{%
}{%
  Erstmals im Jahr 2023 haben große Zahlen neuer Userkonten haben große Mengen Daten in
  OpenStreetMap in Kriegsgebieten kaputt editiert. Vorher hatte es nur Streit zwischen einzelnen
  Mappern oder gut gemeinte missglückte Massendits gegeben.
  Bisher hat die Overpass API ihre Funktionalität daran orientiert, unbeherrschten Mappern
  keinesfalls Werkzeuge für Massenedits anzubieten. Exisiterende Möglichkeiten gegen Vandalismus
  werden gezeigt und zukünftige Features der Overpass API erwogen.%
}%


%%%%%%%%%%%%%%%%%%%%%%%%%%%%%%%%%%%%%%%%%%%

% time: Thursday 16:45
% URL: https://pretalx.com/fossgis2023/talk/fossgis2024-39044-geflschte-papiere-daten-verflschen-um-eine-richtige-print-karte-zu-erzeugen/

%

\noindent\abstractHSvier{%
  Wolfgang Hinsch%
}{%
  Gefälschte Papiere~-- Daten verfälschen, um eine richtige Print-Karte zu erzeugen%
}{%
}{%
  Eine gedruckte Karte unterscheidet sich in sehr vielen Punkten, die nicht immer offensichtlich
  sind, von einer Online-Karte. Häufig müssen dafür die Daten manipuliert werden, und nicht immer
  ist das automatisch möglich.%
}%


%%%%%%%%%%%%%%%%%%%%%%%%%%%%%%%%%%%%%%%%%%%

% time: Thursday 16:45
% URL: https://pretalx.com/fossgis2023/talk/fossgis2024-38021-ask-me-anything-stac-spatiotemporal-asset-catalog-/

%

\noindent\abstractExp{%
  Matthias Mohr%
}{%
  Ask me anything: STAC (SpatioTemporal Asset Catalog)%
}{%
}{%
  STAC eine gemeinsame Sprache zur Beschreibung von Geodaten, so dass diese leichter bearbeitet,
  indiziert und gefunden werden können. Spezifkationen und Ökosystem haben mittlerweile einen
  immensen Umfang. Haben Sie Fragen? Gerne versuchen wir diese zu beantworten...%
}%


%%%%%%%%%%%%%%%%%%%%%%%%%%%%%%%%%%%%%%%%%%%

% time: Thursday 16:45
% URL: https://pretalx.com/fossgis2023/talk/fossgis2024-38770-postnas-suite-anwendertreffen/

%

\noindent\abstractAnwBoFeins{%
  Astrid Emde%
}{%
  PostNAS-Suite Anwendertreffen%
}{%
}{%
  Die PostNAS-Suite Anwender:innen kommunizieren über die Mailingliste und treffen sich zum
  Austausch. Das nächste Treffen soll auf der FOSSGIS 2024 stattfinden. Hier sollen aktuelle
  Entwicklungen im PostNAS-Suite Projekt vorgestellt und die Erfahrungen der Anwender:innen
  ausgetauscht werden.%
}%


%%%%%%%%%%%%%%%%%%%%%%%%%%%%%%%%%%%%%%%%%%%

% time: Thursday 17:20
% URL: https://pretalx.com/fossgis2023/talk/fossgis2024-38690-klimatische-zeitreihenanalyse-zur-modellierung-von-koregionen/

%
\newTimeslot{17:20}
\noindent\abstractHSeins{%
  Markus Metz%
}{%
  Klimatische Zeitreihenanalyse zur Modellierung von Ökoregionen%
}{%
}{%
  Das Ziel dieser Studie war die Identifikation von Gebieten in Nordafrika, die in Bezug auf
  Überträger neuartiger Krankheiten besonders überwacht werden sollten. Dazu wurden Zeitreihen
  verschiedener Fernerkundungsdaten zu Umwelt und Klima aufbereitet, mit deren Hilfe Ökoregionen
  klassifiziert und identifiziert werden können. Mit einer vergleichenden Risikoanalyse konnten
  anschließend entomologische Überwachungsprogramme angepasst und optimiert werden.%
}%


%%%%%%%%%%%%%%%%%%%%%%%%%%%%%%%%%%%%%%%%%%%

% time: Thursday 17:20
% URL: https://pretalx.com/fossgis2023/talk/fossgis2024-38735-qgis-web-client-2-qwc2-neues-aus-dem-projekt/

%

\noindent\abstractHSzwei{%
  Sandro Mani%
}{%
  QGIS Web Client 2 (QWC2)~-- Neues aus dem Projekt%
}{%
}{%
  Dieser Vortrag stellt den QWC2 vor und zeigt, wie einfach es ist, eigene QGIS-Projekte im Web zu
  veröffentlichen. Es wird ein Überblick über die QWC2-Architektur gegeben. Dabei ist es auch eine
  Gelegenheit, die letzten neuen Funktionen, die im letzten Jahr entwickelt wurden, und die Ideen
  für zukünftige Verbesserungen zu entdecken.%
}%


%%%%%%%%%%%%%%%%%%%%%%%%%%%%%%%%%%%%%%%%%%%

% time: Thursday 17:20
% URL: https://pretalx.com/fossgis2023/talk/fossgis2024-39114-malen-nach-zahlen-landnutzungserfassung-in-openstreetmap-in-deutschland/

%
\newpage
\noindent\abstractHSdrei{%
  Michael Reichert%
}{%
  Malen nach Zahlen~--  Landnutzungserfassung in OpenStreetMap in Deutschland%
}{%
}{%
  Der Vortrag widmet sich einer Bestandsaufnahme der Land\-nut\-zungs-Erfassung in OpenStreetMap in
  Deutschland mit Schwerpunkt auf den verschiedenen Erfassungsmethoden. Er beantwortet folgende
  Fragen: Welcher Erfassungsstil (Trennen, Kleben, intensiver Multipolygon-Gebrauch) in dominiert?
  Gibt es regionale Unterschiede? Wie alt sind die Landnutzungsflächen? Wie viel Fläche ist doppelt
  erfasst, wie fragmentiert sind die Flächen?%
}%


%%%%%%%%%%%%%%%%%%%%%%%%%%%%%%%%%%%%%%%%%%%

% time: Thursday 17:20
% URL: https://pretalx.com/fossgis2023/talk/fossgis2024-38557-cartohack-live-osm-basierten-karte-mit-qgis-und-postgis-erstellen/

%

\noindent\abstractHSvier{%
  Mathias Gröbe%
}{%
  CartoHack Live: OSM-basierten Karte mit QGIS und PostGIS erstellen%
}{%
}{%
  Die Arbeitsschritte für die Herstellung einer typischen OSM-ba\-sier\-te Karte im Maßstab 1:10.000 bis
  1:100.000 gleichen sich im Wesentlichen: Es müssen Daten von OpenStreetMap aufbereitet,
  generalisiert und signaturiert werden. Das Projekt "`Graubrot"' bietet hierfür eine Ausgangsbasis
  mit einer Konfiguration für osm2pgsql für den Datenimport, einem Datenschema in PostgreSQL/PostGIS
  und einer ersten Visualisierung in QGIS.%
}%


%%%%%%%%%%%%%%%%%%%%%%%%%%%%%%%%%%%%%%%%%%%

% time: Thursday 17:55
% URL: https://pretalx.com/fossgis2023/talk/fossgis2024-38928-geographyforfuture-mit-geodaten-politik-machen/

%
\newTimeslot{17:55}
\noindent\abstractHSeins{%
  Jannick-J. Klitzschmüller%
}{%
  GeographyForFuture: Mit Geodaten Politik machen%
}{%
}{%
  Geodaten und Karten sind die Basis für politische und gesellschaftliche Entscheidungen. Sie
  produzieren und reproduzieren soziale Wirklichkeiten und lenken unser alltägliches Handeln. Aber
  wir können Geodaten auch für den Kampf für eine bessere Welt nutzen. Woher die Macht von Geodaten
  kommt und wie wir Geodaten politisch nutzen können, sind die Themen des Talks.%
}%


%%%%%%%%%%%%%%%%%%%%%%%%%%%%%%%%%%%%%%%%%%%

% time: Thursday 17:55
% URL: https://pretalx.com/fossgis2023/talk/fossgis2024-38932-ein-wanderwegegis-fr-den-sauerlndischen-gebirgs-und-wanderverein-/

%

\noindent\abstractHSzwei{%
  Claas Leiner%
}{%
  WanderwegeGIS für den Sauerländischen Gebirgs- u. Wanderverein.%
}{%
}{%
  Der Sauerländischen Gebirgs- und Wanderverein betreut ein Wegenetz mit über 4000 Routen in NRW.
  Umgesetzt mit QGIS und PostGis, können Ehren- und Hauptamtliche jetzt gemeinsam auf den
  Datenbestand zugreifen, Wege editieren, umbennenn und als Tracks exportieren. Auf jedem
  Trassenabschnitt sind die Wegesysmbole der jeweiligen Routen zu sehen.   Über ein Python-Plugin
  werden komplexe  Prozesse  einfach umsetzbar.%
}%


%%%%%%%%%%%%%%%%%%%%%%%%%%%%%%%%%%%%%%%%%%%

% time: Thursday 19:00
% URL: https://pretalx.com/fossgis2023/talk/fossgis2024-39679-mitgliederversammlung/

%
\newSmallTimeslot{19:00}
\noindent\abstractHSzwei{%
  FOSSGIS e.V.%
}{%
  Mitgliederversammlung%
}{%
}{%
  Zur jährlich stattfindenden Versammlung des FOSSGIS e.V. sind alle Mitglieder herzlich eingeladen.  Der FOSSGIS e.V. lädt ein zum Kennenlernen, zur Diskussion,
  Abstimmung und Wahlen.%
}%


%%%%%%%%%%%%%%%%%%%%%%%%%%%%%%%%%%%%%%%%%%%
