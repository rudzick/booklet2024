
% time: Wednesday 10:00
% URL: https://pretalx.com/fossgis2023/talk/fossgis2024-39676-erffnung/

%
\newSmallTimeslot{10:00}
\noindent\abstractHSeins{%
  FOSSGIS e.V.%
}{%
  Eröffnung%
}{%
}{%
  Feierliche Eröffnung der Konferenz durch Vertreter des FOSSGIS e.V. mit wertvollen Hinweisen zum
  Ablauf und der Organisation.%
}%


%%%%%%%%%%%%%%%%%%%%%%%%%%%%%%%%%%%%%%%%%%%

% time: Wednesday 10:50
% URL: https://pretalx.com/fossgis2023/talk/fossgis2024-44186-einsatz-von-open-data-und-open-source-in-hh-eine-erfolgsversprechende-strategie-/

%
\newSmallTimeslot{10:50}
\noindent\abstractHSeins{%
  Thomas Eichhorn%
}{%
  Einsatz von Open Data und Open Source in HH~-- Eine erfolg\-versprechende Strategie!%
}{%
}{%
  Der Geschäftsführer des Landesbetrieb Geoinformation und Vermessung in Hamburg, Thomas Eichhorn,
  gibt einen Einblick wie Open Data und Open Source in Hamburg im Bereich der Geoinformationen
  eingesetzt wird.%
}%

\small
\sponsorBoxA{201_HBT.png}{0.45\textwidth}{6}{%
\textbf{Silbersponsor und Aussteller}%
\vspace{0.5\baselineskip}

\noindent
{\bfseries HBT Hamburger Berater Team GmbH} HBT ist der strategische Entwicklungspartner der Hamburger HOCHBAHN, gemeinsam setzen wir von Anfang an konsequent auf Open-Source-Software.

\noindent
Entstanden ist eine moderne und leistungsstarke Software, die das Rückgrat für sämtliche Auskünfte rund um das ÖPNV-Netz der Metropolregion Hamburg bildet – operated in der Public Cloud. Open Source-Projekte wie u.a. OpenStreetMap und GeoServer bilden die Basis. Aktuell wird OpenTripPlanner (OTP) als Kern einer neuen und verbesserten multimodalen Fahrplanauskunft evaluiert. OTP wird in Zukunft die aktuelle Routing-Engine ergänzen.
}%
\normalsize


%%%%%%%%%%%%%%%%%%%%%%%%%%%%%%%%%%%%%%%%%%%

% time: Wednesday 11:15
% URL: https://pretalx.com/fossgis2023/talk/fossgis2024-39651-sovereign-cloud-stack-scs-offene-fderierbare-cloud-technologie-fr-jeden-sektor/

%
\newTimeslot{11:15}
\noindent\abstractHSeins{%
  Manuela Urban%
}{%
  Sovereign Cloud Stack (SCS): Offene, föderierbare Cloud-Technologie für jeden Sektor%
}{%
}{%
  Die zertifizierbaren offenen Standards und eine OS-Referenz\-implementierung für einen
  vollständigen, modularen Cloud- und Container Stack der Sovereign Cloud Stack (SCS) Community
  ermöglichen auch kleineren und mittleren Providern sowie internen IT-Dienstleistern
  State-of-the-art-Technologie einzusetzen. Die Standards ermöglichen außerdem, Cloud-Ressourcen
  organisationsübergreifend zu föderieren und somit gemeinschaftlich und dezentral ein
  leistungsstarkes "`Sovereign Cloud Grid"' zu bilden.%
}%

\sponsorBoxA{402_terrestris.png}{0.45\textwidth}{5}{%
\textbf{Bronzesponsor und Aussteller}\\
\noindent\small {\bfseries terrestris GmbH \& Co. KG} terrestris ist Dienstleister für maßgeschneiderte Geoinformations-Lösungen mit Freier \& Open Source Software und deckt das gesamte Spektrum von Beratung, Konzeptionierung, Entwicklung bis hin zu Wartung \& Support ab. Wir entwickeln Lösungen, die den tatsächlichen Anforderungen unserer Kunden entsprechen.
\normalsize
}%


%%%%%%%%%%%%%%%%%%%%%%%%%%%%%%%%%%%%%%%%%%%

% time: Wednesday 11:45
% URL: https://pretalx.com/fossgis2023/talk/fossgis2024-38834-automatisierte-bestimmung-der-straenbeschaffenheit-mit-machine-learning/

%
\newTimeslot{11:45}
\noindent\abstractHSeins{%
  Alexandra Kapp, Edith Hoffmann%
}{%
  Automatisierte Bestimmung der Straßenbeschaffenheit mit Machine Learning%
}{%
}{%
  Flächendeckende Daten zu Straßenbeschaffenheit in einem einheitlichen Format wären für Routing
  oder Stadtplanung eine hilfreiche Information.
  Das mFund Projekt "`SurfaceAI"' hat sich zum Ziel gesetzt, auf offenen Daten ein Machine Learning
  Modell zu trainieren, das den Belag und die Qualität der Straßenoberfläche anhand eines Fotos mit
  hoher Genauigkeit erkennt. Das Modell bildet die Grundlage, um Straßenbilder mit georeferenzierten
  Datensätzen auf Straßenebene zu verknüpfen.%
}%


%%%%%%%%%%%%%%%%%%%%%%%%%%%%%%%%%%%%%%%%%%%

% time: Wednesday 11:45
% URL: https://pretalx.com/fossgis2023/talk/fossgis2024-38019-neues-in-stac/

%

\noindent\abstractHSzwei{%
  Matthias Mohr%
}{%
  Neues in STAC%
}{%
}{%
  Die STAC-Spezifikationen (SpatioTemporal Asset Catalog) sind eine flexible Sprache zur
  Beschreibung von Geodaten in verschiedenen Bereichen und für eine Vielzahl von Anwendungsfällen.
  In diesem Vortrag wird der aktuelle Stand der Spezifikationen vorgestellt, zu denen die
  STAC-Spezifikation für statische Kataloge und die auf OGC-APIs aufbauende API-Spezifikation
  gehören. Hierbei wird ein Fokus darauf liegem, was neu in STAC Version 1.1 enthalten ist. Zudem
  gibt es Informationen zu den Änderungen%
}%


%%%%%%%%%%%%%%%%%%%%%%%%%%%%%%%%%%%%%%%%%%%

% time: Wednesday 11:45
% URL: https://pretalx.com/fossgis2023/talk/fossgis2024-38994-stand-des-grass-gis-projekts-nicht-was-sie-denken-/

%

\noindent\abstractHSdrei{%
  Markus Neteler%
}{%
  Stand des GRASS GIS Projekts: Nicht was Sie denken!%
}{%
}{%
  In unserem Vortrag geben wir einen Überblick über die neuesten Entwicklungen und Fortschritte des
  GRASS GIS Projekts, das im Sommer 2023 sein 40-jähriges Jubiläum feierte. Der Fokus liegt darauf,
  Missverständnisse rund um das Projekt aufzuklären und die tatsächliche Vielseitigkeit und
  Modernität der aktuellen GRASS GIS Version zu verdeutlichen. Der Vortrag wird einige häufige
  Missverständnisse ausräumen, wie z.B. "`es ist nur eine Kommandozeile"', "`es ist nur ein Desktop
  GIS"' und mehr.%
}%


%%%%%%%%%%%%%%%%%%%%%%%%%%%%%%%%%%%%%%%%%%%

% time: Wednesday 11:45
% URL: https://pretalx.com/fossgis2023/talk/fossgis2024-38772-mapbender-die-neue-version-4-stellt-sich-vor/

%

\noindent\abstractHSvier{%
  Astrid Emde%
}{%
  Mapbender~-- die neue Version 4 stellt sich vor%
}{%
}{%
  In diesem Vortrag soll der Umgang mit dem WebGIS Client Mapbender demonstriert werden.
  Mapbender bietet die Möglichkeit eine unbegrenzte Anzahl von Anwendungen zu erzeugen. Die
  Anwendungen können nach Belieben aufgebaut und mit Kartendiensten ausgestattet werden. Es können
  leicht individuelle Suchen und Datenerfassung aufgebaut werden. Dies erfolgt alles ohne Code
  schreiben zu müssen.%
}%


%%%%%%%%%%%%%%%%%%%%%%%%%%%%%%%%%%%%%%%%%%%

% time: Wednesday 12:20
% URL: https://pretalx.com/fossgis2023/talk/fossgis2024-38990-der-weg-eines-schlaglochs-von-der-strae-auf-die-karte/

%
\newTimeslot{12:20}
\noindent\abstractHSeins{%
  Asmus Harder, Melanie Fleischer%
}{%
  Der Weg eines Schlaglochs von der Straße auf die Karte%
}{%
}{%
  Jeder und jede von uns hat sich sicher schon einmal darüber geärgert: nichts Böses ahnend bleibt
  der Fuß oder das Rad in einem Schlagloch hängen, dass so tief ist, dass es fast auf die andere
  Seite der Erde reicht. Doch welchen Weg durch die GIS-Welt geht ein Schlagloch bis es auf einer
  Karte in der zuständigen Verwaltung landet? Und welche Rolle spielen hier Freie und Open-Source
  Software?%
}%


%%%%%%%%%%%%%%%%%%%%%%%%%%%%%%%%%%%%%%%%%%%

% time: Wednesday 12:20
% URL: https://pretalx.com/fossgis2023/talk/fossgis2024-38982-bereitstellung-von-freien-geodaten-opendata-mit-stac-beim-lgln/

%

\noindent\abstractHSzwei{%
  Katrin Pinkert, Ralf Wohlfahrt%
}{%
  Bereitstellung von freien Geodaten (OpenData) mit STAC beim LGLN%
}{%
}{%
  Es wird der Einsatz von STAC (SpatioTemporal Asset Catalog) zur Katalogisierung von freien
  Geodaten (OpenData) beim Landesamt für Geoinformation und Landesvermessung Niedersachsen (LGLN)
  auf Basis von OSS vorgestellt.%
}%

%%%%%%%%%%%%%%%%%%%%%%%%%%%%%%%%%%%%%%%%%%%

% time: Wednesday 12:20
% URL: https://pretalx.com/fossgis2023/talk/fossgis2024-39040-wie-schreibe-ich-ein-grass-gis-addon-/

%
\newpage
\noindent\abstractHSdrei{%
  Carmen Tawalika, Markus Neteler%
}{%
  Wie schreibe ich ein GRASS GIS Addon?%
}{%
}{%
  In diesem Vortrag werden bewährte Praktiken beim Entwickeln von GRASS GIS Addons vorgestellt.
  Neben der Bibliothek "`grass-gis-helpers"', die Werkzeuge und Hilfsmittel bereitstellt, wird auch
  die Nutzung wiederverwendbarer Workflows gezeigt, um durch automatisiertes Linten und Testen die
  Qualität des Codes sicherzustellen.
  Abschließend werden einige von uns entwickelte GRASS GIS Addons vorgestellt.
  Lassen Sie sich inspirieren, um danach vielleicht ein eigenes GRASS GIS Addon zu entwickeln!%
}%


%%%%%%%%%%%%%%%%%%%%%%%%%%%%%%%%%%%%%%%%%%%

% time: Wednesday 12:20
% URL: https://pretalx.com/fossgis2023/talk/fossgis2024-39039-pyqgis-schnuppervortrag-mein-erstes-plugin-fr-qgis/

%

\noindent\abstractHSvier{%
  Gordon Schlolaut%
}{%
  PyQGIS Schnuppervortrag~-- Mein erstes Plugin für QGIS%
}{%
}{%
  Wie erstelle ich mein erstes QGIS-Plugin? Eine Live-Demo für alle, die schon immer wissen wollten,
  wie es geht.%
}%

\sponsorBoxA{403_mundialis.png}{0.45\textwidth}{4}{%
\textbf{Bronzesponsor}\\
\noindent\small {\bfseries mundialis GmbH \& Co. KG} mundialis ist spezialisiert auf die Auswertung und Verarbeitung von Fernerkundungs- und Geodaten mit dem Schwerpunkt Cloud-basierte Geoprozessierung. Wir setzen Freie \& Open Source Geoinformationssysteme (GRASS GIS, actinia, QGIS, u.a.) ein, mit denen wir maßgeschneiderte Lösungen für den Kunden entwickeln.
\normalsize
}%


%%%%%%%%%%%%%%%%%%%%%%%%%%%%%%%%%%%%%%%%%%%

% time: Wednesday 14:15
% URL: https://pretalx.com/fossgis2023/talk/fossgis2024-39705-studierende-stellen-ihre-arbeit-vor/

%
\newTimeslot{14:15}
\noindent\abstractAnwBoFeins{%
  %
}{%
  Studierende stellen Ihre Arbeit vor%
}{%
}{%
  Studierende stellen Ihre Arbeit vor (Masterarbeit, Bachelorarbeit, aktuell in Arbeit,
  Seminararbeit, Praktikumsaufgaben, Abschlussarbeiten(Ausbildung) .%
}%


%%%%%%%%%%%%%%%%%%%%%%%%%%%%%%%%%%%%%%%%%%%

% time: Wednesday 14:15
% URL: https://pretalx.com/fossgis2023/talk/fossgis2024-38916-lizmap-webclient/

%

\noindent\abstractExp{%
  Günter Wagner%
}{%
  Lizmap Webclient%
}{%
}{%
  Diese Fragestunde, kombiniert mit Demo-Beispielen, ermöglicht Interessierten einen Einblick in den
  Webclient Lizmap. Es werden spezielle Funktionen und Neuerungen vorgestellt, die ggf. auch für
  Anwender interessant, bzw. neu sind. Der Schwerpunkt liegt bei Fragen der Teilnehmer.%
}%


%%%%%%%%%%%%%%%%%%%%%%%%%%%%%%%%%%%%%%%%%%%

% time: Wednesday 14:15
% URL: https://pretalx.com/fossgis2023/talk/fossgis2024-38058-open-data-greenpeace-e-v-/

%

\noindent\abstractHSeins{%
  Jonathan Niesel%
}{%
  Open Data @Greenpeace e.V.%
}{%
}{%
  Es wird über die Einführung eines offenen Datenportals von Greenpeace Deutschland berichtet
  (https://daten.greenpeace.de/).
  Dabei wird aus der Praxis berichtet : welche Software wurde warum ausgewählt , welche
  (un)erwarteten Hürden gab es bei der Einführung/Entwicklung%
}%


%%%%%%%%%%%%%%%%%%%%%%%%%%%%%%%%%%%%%%%%%%%

% time: Wednesday 14:15
% URL: https://pretalx.com/fossgis2023/talk/fossgis2024-38758-bbox-kompakter-ogc-api-server-fr-features-tiles-und-mehr/

%
\newpage
\noindent\abstractHSzwei{%
  Pirmin Kalberer%
}{%
  BBOX: Kompakter OGC API Server für Features, Tiles und mehr%
}{%
}{%
  [BBOX](https://sourcepole.github.io/bbox/) ist eine Open Source Implementation der neuen OGC API
  Standards mit Rückwärtskompatiblität zu WMS und WFS. Rasterkarten werden mit UMN MapServer oder
  QGIS Server gerendert, Vektorkacheln können direkt aus PostGIS-Daten erzeugt werden.
  BBOX ist in der Programmiersprache Rust implementiert und enthält einen hochperformanten
  Web-Server.%
}%


%%%%%%%%%%%%%%%%%%%%%%%%%%%%%%%%%%%%%%%%%%%

% time: Wednesday 14:15
% URL: https://pretalx.com/fossgis2023/talk/fossgis2024-38831-aufbau-einer-agrar-forschungsdateninfrastruktur-mit-geonode-und-kubernetes/

%

\noindent\abstractHSdrei{%
  Marcel Wallschläger%
}{%
  Aufbau einer Agrar-Forschungsdateninfrastruktur mit GeoNode und Kubernetes%
}{%
}{%
  In diesem Vortrag werde ich das zukünftige Daten Repositorium unserer  Arbeitsgruppe für
  Forschungsdatenmanagement am ZALF vorstellen.  Ich werde das Paradigma "`Infrastructure-as-Code"
  praktisch erläutern. Darauf aufbauend die Bedeutung unserer Kubernetes Cluster erklären sowie
  Tools wie ArgoCD vorstellen.
  Anschließend werde ich auf unsere GeoNode Installation auf Kubernetes eingehen.
  - https://github.com/zalf-rdm/geonode
  - https://github.com/zalf-rdm/geonode-k8s%
}%


%%%%%%%%%%%%%%%%%%%%%%%%%%%%%%%%%%%%%%%%%%%

% time: Wednesday 14:15
% URL: https://pretalx.com/fossgis2023/talk/fossgis2024-38903-xplanbox-kommunale-datendrehscheibe-fr-die-bauleit-und-landschaftsplanung/

%

\noindent\abstractHSvier{%
  Stefan Peuser%
}{%
  xPlanBox: kommunale Datendrehscheibe für die Bauleit- und Landschaftsplanung%
}{%
}{%
  XPlanung ist der verbindliche Standard für die Bauleit- und Landschaftsplanung auf kommunaler
  Ebene. Die OpenSource-An\-wen\-dung xPlanBox unterstützt die Kommunen als zentrales
  Managementverfahren für sämtliche Planwerke sowie als Bereitstellungsplattform von XPlanung-Daten.
  Durch Containerlösungen ist der Betrieb für mehrere Mandanten einfach möglich, ebenso der Zugriff
  auf die Daten durch die Integration im Masterportal.%
}%
\bigskip

\sponsorBoxA{404_52North.png}{0.45\textwidth}{4}{%
\textbf{Bronzesponsor}\\
\noindent\small {\bfseries 52°North Spatial Information Research GmbH} 52°North ist ein angewandtes Forschungsunternehmen im Bereich der Geo-Dateninfrastrukturen und der Ableitung von räumlichen Informationsprodukten. Als non-profit Unternehmen unterstützt 52°North Open Science durch offene Daten und Open-Source-Software. Unser Hauptinteresse gilt der Entwicklung räumlicher Forschungsdateninfrastrukturen, um die Ableitung von Informationen aus Daten zu erleichtern und zu fördern.
\normalsize
}%


%%%%%%%%%%%%%%%%%%%%%%%%%%%%%%%%%%%%%%%%%%%

% time: Wednesday 14:50
% URL: https://pretalx.com/fossgis2023/talk/fossgis2024-39029-hinweiskarten-starkregengefahren-opendata-fr-die-bundesweite-klimawandelanpassung/

%
\newTimeslot{14:50}
\noindent\abstractHSeins{%
  Lukas Wimmer%
}{%
  Hinweiskarten Starkregengefahren: OpenData für die bundesweite Klimawandelanpassung%
}{%
}{%
  Infolge des Klimawandels treten Starkregenereignisse häufiger und intensiver auf; sie verursachen
  jährlich erhebliche Schäden. Als Beitrag zu einer optimalen staatlichen Vorsorge erstellt das
  Bundesamt für Kartographie und Geodäsie (BKG) eine bundesweite und einheitliche Hinweiskarte zu
  Starkregengefahren. Diese wird bis Ende 2025 sukzessive erstellt und Entscheidungsträgern, dem
  Katastrophenschutz sowie der gesamten Öffentlichkeit als OpenData frei zugänglich gemacht.%
}%


%%%%%%%%%%%%%%%%%%%%%%%%%%%%%%%%%%%%%%%%%%%

% time: Wednesday 14:50
% URL: https://pretalx.com/fossgis2023/talk/fossgis2024-38773-pygeoapi-eine-python-server-software-fr-ogc-api-standards/

%

\noindent\abstractHSzwei{%
  Astrid Emde%
}{%
  pygeoapi~-- eine Python Server Software für OGC API Standards%
}{%
}{%
  Lernen Sie die Python Server Implementation der OGC API suite of standards kennen.
  Anhand von einfachen Beispielen wird vorgeführt wie mit pygeoapi OGC API Dienste erstellt werden
  können.%
}%


%%%%%%%%%%%%%%%%%%%%%%%%%%%%%%%%%%%%%%%%%%%

% time: Wednesday 14:50
% URL: https://pretalx.com/fossgis2023/talk/fossgis2024-38979-geonetwork-ui-ein-anwenderfreundliches-frontend-fr-den-datenkatalog-geonetwork/

%
\newpage
\noindent\abstractHSdrei{%
  Angelika Kinas%
}{%
  GeoNetwork-UI: Ein anwenderfreundliches Frontend für den Datenkatalog GeoNetwork%
}{%
}{%
  Das Open-Source Projekt GeoNetwork-UI steht in enger Verbindung zur klassischen
  Metadatenkatalog-Anwendung GeoNetwork.
  In dieser Präsentation werden wir sowohl den aktuellen Stand des Projekts GeoNetwork-UI, das
  innovative Design des neuen Metadaten-Editors, als auch die bevorstehenden Entwicklungen
  vorstellen.%
}%


%%%%%%%%%%%%%%%%%%%%%%%%%%%%%%%%%%%%%%%%%%%

% time: Wednesday 14:50
% URL: https://pretalx.com/fossgis2023/talk/fossgis2024-38732-xplanung-fr-die-cloud/

%

\noindent\abstractHSvier{%
  Torsten Friebe%
}{%
  XPlanung für die Cloud%
}{%
}{%
  Im April 2022 wurde der Quellcode der Software xPlanBox der Firma lat/lon im Rahmen eines
  Pilotprojekts auf der OpenCoDE-Plattform des BMI veröffentlicht. Seitdem wird die Software
  kontinuierlich weiterentwickelt und kommt im Rahmen des Onlinezugangsgesetz (OZG) und des
  "`Einer-für-Alle"-Prinzips (EfA) zum Einsatz. Der Vortrag stellt kurz die wichtigsten Erweiterung
  der Software für den Betrieb in der Cloud vor.%
}%


%%%%%%%%%%%%%%%%%%%%%%%%%%%%%%%%%%%%%%%%%%%

% time: Wednesday 15:25
% URL: https://pretalx.com/fossgis2023/talk/fossgis2024-38883-starkregengefahrenhinweiskarten-fr-niedersachsen-schleswig-holstein-hb-und-hamburg/

%
\newTimeslot{15:25}
\noindent\abstractHSeins{%
  Barbara Werth, Uwe Ross%
}{%
  Starkregengefahrenhinweiskarten für Niedersachsen/Schleswig-Holstein/HB und Hamburg%
}{%
}{%
  Die Wissenschaft ist sich einig, dass er fortschreitende Klimawandel zu einer Zunahme von
  Extremwetterereignissen (u.a. Starkregen) führt. Das Bundesamt für Kartografie und Geodäsie hat
  das Ziel, eine bundesweit einheitliche Starkregengefahrenhinweiskarte zu erstellen. Die
  Arbeitsgemeinschaft (Fischer Teamplan und Weber–Ingenieure) erstellt derzeit mit
  OpenSource-Software (HiPIMS und QGIS) und überwiegend mit OpenData diese Karten für
  Niedersachsen/Schleswig-Holstein/Bremen und Hamburg.%
}%


%%%%%%%%%%%%%%%%%%%%%%%%%%%%%%%%%%%%%%%%%%%

% time: Wednesday 15:25
% URL: https://pretalx.com/fossgis2023/talk/fossgis2024-38343-pgfeatureserv-verffentlichung-von-vektordaten-mit-ogc-api-features/

%

\noindent\abstractHSzwei{%
  Jakob Miksch%
}{%
  pg\_featureserv~-- Veröffentlichung von Vektordaten mit OGC API Features%
}{%
}{%
  Im Vortrag wird pg\_featureserv vorgestellt, ein leichtgewichtiges Programm zur Veröffentlichung
  von Vektordaten aus einer PostGIS Datenbank über den OGC API Features Standard. Es wird die
  Installation erklärt, die verschiedenen Filtermöglichkeiten der Daten besprochen und es wird
  genauer auf die Eigenschaften des OGC API Features Standards als Nachfolger des WFS (Web Feature
  Service) eingegangen.%
}%


%%%%%%%%%%%%%%%%%%%%%%%%%%%%%%%%%%%%%%%%%%%

% time: Wednesday 15:25
% URL: https://pretalx.com/fossgis2023/talk/fossgis2024-38423-geoserver-cloud-mit-kubernetes/

%

\noindent\abstractHSdrei{%
  Nils Bühner%
}{%
  GeoServer Cloud mit Kubernetes%
}{%
}{%
  "`GeoServer Cloud"' ist ein Projekt, welches das Ziel verfolgt die vom klassischen GeoServer
  implementierten OGC-Standards und weitere Schnittstellen als individuell skalierbare Microservices
  "`cloud native"' in einer Container-basierten Umgebung bereitzustellen. Der Vortrag stellt die
  Möglichkeiten der (OGC-)Dienst-Orchestrierung am Beispiel von Kubernetes vor und beleuchtet die
  Vor- und Nachteile dieser Entwicklungen.%
}%


%%%%%%%%%%%%%%%%%%%%%%%%%%%%%%%%%%%%%%%%%%%

% time: Wednesday 15:25
% URL: https://pretalx.com/fossgis2023/talk/fossgis2024-38908-einsatz-von-machine-learning-zur-erstellung-von-xplangml/

%

\noindent\abstractHSvier{%
  Julian Zilz%
}{%
  Einsatz von Machine Learning zur Erstellung von XPlanGML%
}{%
}{%
  Der Vortrag präsentiert einen methodischen Ansatz zur automatisierten Erstellung von XPlanGML im
  Raster-Umring-Szenario mithilfe von Machine Learning. Es werden die einzelnen Schritte des
  Workflows sowie die Einsatzmöglichkeiten von Machine Learning im Detail erläutert.%
}%


%%%%%%%%%%%%%%%%%%%%%%%%%%%%%%%%%%%%%%%%%%%

% time: Wednesday 16:30
% URL: https://pretalx.com/fossgis2023/talk/fossgis2024-39704-studierende-stellen-ihre-arbeit-vor/

%
\newTimeslot{16:30}
\noindent\abstractAnwBoFeins{%
  %
}{%
  Studierende stellen Ihre Arbeit vor%
}{%
}{%
  Studierende stellen Ihre Arbeit vor (Masterarbeit, Bachelorarbeit, aktuell in Arbeit,
  Seminararbeit, Praktikumsaufgaben, Abschlussarbeiten(Ausbildung) .%
}%


%%%%%%%%%%%%%%%%%%%%%%%%%%%%%%%%%%%%%%%%%%%

% time: Wednesday 16:30
% URL: https://pretalx.com/fossgis2023/talk/fossgis2024-38912-anwendertreffen-lizmap-webclient/

%

\noindent\abstractAnwBoFzwei{%
  Günter Wagner%
}{%
  Anwendertreffen Lizmap-Webclient%
}{%
}{%
  Die deutschsprachige Anwendergruppe für den WebClient Lizmap möchte das Treffen zum
  Erfahrungsaustausch nutzen.
  Teilnehmer können ihre eigenen, mit Lizmap realisierten, WebGIS-Projekte vorstellen. Ferner kann
  über aktuelle Fragen/Probleme und zukünftige, gewünschte Erweiterungen in Lizmap diskutiert
  werden.
  Das Anwendertreffen richtet sich sowohl an neu Interessierte, als auch an Anwender, die bereits
  mit Lizmap arbeiten.%
}%


%%%%%%%%%%%%%%%%%%%%%%%%%%%%%%%%%%%%%%%%%%%

% time: Wednesday 16:30
% URL: https://pretalx.com/fossgis2023/talk/fossgis2024-39009-ask-me-anything-qgis-/

%

\noindent\abstractExp{%
  Marco Bernasocchi%
}{%
  Ask me anything QGIS!%
}{%
}{%
  QGIS-Chairman Marco Bernasocchi und Kernentwickler Matthias Kuhn stehen während einer Stunde für
  alle QGIS-relevanten Fragen zur Verfügung.%
}%


%%%%%%%%%%%%%%%%%%%%%%%%%%%%%%%%%%%%%%%%%%%

% time: Wednesday 16:30
% URL: https://pretalx.com/fossgis2023/talk/fossgis2024-39002-prozedurale-kunst-mit-qgis/

%
\newpage
\noindent\abstractHSeins{%
  Johannes Kröger%
}{%
  Prozedurale Kunst mit QGIS%
}{%
}{%
  Mit QGIS als Leinwand, dem Geometrie-Generator als Pinsel und datendefinierter Übersteuerung als
  Palette wird gemalt und animiert bis die CPU bricht.%
}%

%%%%%%%%%%%%%%%%%%%%%%%%%%%%%%%%%%%%%%%%%%%

% time: Wednesday 16:35
% URL: https://pretalx.com/fossgis2023/talk/fossgis2024-38822-geostyler-ein-visueller-vergleich/

%
\newSmallTimeslot{16:35}
\noindent\abstractHSeins{%
  Jan Suleiman%
}{%
  GeoStyler~-- Ein visueller Vergleich%
}{%
}{%
  In diesem Talk führen wir einen visuellen Vergleich der von GeoStyler unterstützten Stilformate
  anhand einiger Beispielkarten durch.%
}%


%%%%%%%%%%%%%%%%%%%%%%%%%%%%%%%%%%%%%%%%%%%

% time: Wednesday 16:40
% URL: https://pretalx.com/fossgis2023/talk/fossgis2024-38829-beyond-webgis-empowering-scrollytelling-with-maps-and-data/

%
\newSmallTimeslot{16:40}
\noindent\abstractHSeins{%
  Hannes Blitza%
}{%
  Beyond WebGIS: Empowering\linebreak Scrollytelling with Maps and Data%
}{%
}{%
  Nicht nur journalistische Formate profitieren von dynamischen multimedialen Scrollytelling~-- ob
  Stadt- oder Windparkplanung, fast jede Geschichte mit einem geographischen Bezug lässt sich durch
  interaktive Karten, Charts oder Tabellen ausgestalten und Leser:innen zu neuen Gedanken
  inspirieren. Wir demonstrieren exemplarisch u.a. mit dem Business Intelligence Tool Apache
  Superset, wie man die zahlreichen Visualisierungsmöglichkeiten des Scollytellings vollends
  ausschöpfen kann.%
}%


%%%%%%%%%%%%%%%%%%%%%%%%%%%%%%%%%%%%%%%%%%%

% time: Wednesday 16:45
% URL: https://pretalx.com/fossgis2023/talk/fossgis2024-38991-3dprojektplaner-stadtentwicklung-mit-perspektive/

%
\newSmallTimeslot{16:45}
\noindent\abstractHSeins{%
  Mateusz Lendziński%
}{%
  3DProjektplaner~-- Stadtentwicklung mit Perspektive%
}{%
}{%
  Der 3DProjektplaner ist eine auf dem Open Source Masterportal basierende Webanwendung, die es
  planenden Dienststellen in der Verwaltung ermöglicht, Bauvorhaben im 3D-Stadtkontext
  geodatenbasiert zu analysieren sowie eigene städtebauliche Entwicklungsideen schnell und einfach
  zu skizzieren. Im Lightning Talk sollen die wesentlichen Funktionen des 3DProjektplaners sowie
  mögliche Anwendungsfälle des Tools aufgezeigt werden.%
}%



%%%%%%%%%%%%%%%%%%%%%%%%%%%%%%%%%%%%%%%%%%%

% time: Wednesday 16:30
% URL: https://pretalx.com/fossgis2023/talk/fossgis2024-38815-weltwrmestrom-datenbank-projekt-webbasierte-explorationswerkzeuge-fr-punktdaten/

%

\noindent\abstractHSzwei{%
  Nikolas Ott%
}{%
  Weltwärmestrom Datenbank\linebreak Projekt: webbasierte Explorationswerkzeuge für Punktdaten%
}{%
}{%
  Im Weltwärmestrom-Datenbank Projekt wird eine neue For\-schungs\-daten\-infra\-struktur für terrestrische
  Wärmestromdaten aufgebaut. Zentraler Aspekt ist die Einhaltung der FAIR und OPEN Datenpolitik.
  Unter anderem werden webbasierte fachbezogene Explora\-tions- und Analysewerkzeuge angeboten, womit
  sich Nutzende bereits im Browser einen ersten Überblick über die Daten verschaffen können. Fokus
  dieser Präsentation ist die technische Umsetzung dieser Explorationswerkzeuge.%
}%


%%%%%%%%%%%%%%%%%%%%%%%%%%%%%%%%%%%%%%%%%%%

% time: Wednesday 16:30
% URL: https://pretalx.com/fossgis2023/talk/fossgis2024-38921-polar-vollkonfigurierbare-pluginbasierte-kartenklienten-fr-brgernahe-anwendungen/

%

\noindent\abstractHSdrei{%
  Pascal Röhling, Dennis Sen%
}{%
  POLAR~-- Vollkonfigurierbare, pluginbasierte Kartenklienten für bürgernahe Anwendungen%
}{%
}{%
  Die Paketbibliothek POLAR wurde Ende 2023 als Open-Source-Projekt auf GitHub veröffentlicht.
  Basierend auf OpenLayers und unter Verwendung von Vue werden verschiedenste wiederverwendbare
  Funktionalitäten publiziert, welche gemeinsam als Kartenklient für zahlreiche Anwendungsgebiete im
  Einsatz sind.
  So verwenden Bürger bereits heute POLAR, u. a. im Meldemichel Hamburg, im
  Denkmalinformationssystem SH, und in einer Vielzahl von Antragssystemen deutscher Behörden.%
}%


%%%%%%%%%%%%%%%%%%%%%%%%%%%%%%%%%%%%%%%%%%%

% time: Wednesday 16:30
% URL: https://pretalx.com/fossgis2023/talk/fossgis2024-38914-modellierung-von-fuzzyness-wobbliness-in-geodaten/

%

\noindent\abstractHSvier{%
  Florian Thiery%
}{%
  Modellierung von Fuzzyness / Wobbliness in Geodaten%
}{%
}{%
  Insbesondere bei der Bereitstellung von Open Data nach den FAIR-Prinzipien zur bestmöglichen
  Offenheit und Transparenz ist die Angabe von Unsicherheiten und Zweifeln für den Nachnutzenden von
  enormer Bedeutung. Bei interdisziplinärer Zusammenarbeit ist dieser Aspekt umso wichtiger. In
  diesem Paper werden fünf data-driven interdisziplinäre Use-Cases für den Umgang mit und die
  Modellierung von vagen und unsicheren Georeferenzen aus dem Bereich der Archäologie und
  Geowissenschaften vorgestellt.%
}%


%%%%%%%%%%%%%%%%%%%%%%%%%%%%%%%%%%%%%%%%%%%

% time: Wednesday 17:05
% URL: https://pretalx.com/fossgis2023/talk/fossgis2024-39032-es-ist-doch-nur-software-zusammenfassung-der-ag-aktivitten/

%
\newTimeslot{17:05}
\noindent\abstractHSeins{%
  Florian Micklich, Torsten Friebe, Torsten Wiebke, David Arndt%
}{%
  Es ist doch nur Software~-- Zusammenfassung der AG-Aktivitäten%
}{%
}{%
  Seit Juni 2021 beschäftigt sich die Arbeitsgruppe ["Öffentliche Ausschreibungen mit FOSS"' des
  FOSSGIS e.V.](https://www.fossgis.de/wiki/AG\_oeffentl\_Ausschreibungen\_FOSS) mit dem Thema der
  Beschaffung und Vergabe von IT-Lösungen auf Basis von FOSS.
  Im Vortrag werden die Aktivitäten und Inhalte, die die AG in Form von zwei Workshops
  durchgeführt hat, zusammengefasst. Es geht um OSS-Lizenzen, rechtliche Rahmenbedingungen und
  technische Sicherheit und sowie Richtlinien.%
}%


%%%%%%%%%%%%%%%%%%%%%%%%%%%%%%%%%%%%%%%%%%%

% time: Wednesday 17:05
% URL: https://pretalx.com/fossgis2023/talk/fossgis2024-39035-monitoring-von-waldgebieten-mit-hilfe-von-sentinel-2-abgeleiteten-vegetationsindizes/

%

\noindent\abstractHSzwei{%
  Markus Eichhorn%
}{%
  Monitoring von Waldgebieten mit Hilfe von Sentinel-2 abgeleiteten Vegetationsindizes%
}{%
}{%
  Innerhalb des Projektes werden Sentinel-2 L2A-Szenen über Irland ausgewählt. Für diese Szenen
  wurden zwei Vegetationsindizes berechnet, sowie mittels zonaler Statistiken der Median dieser
  Vegetationsindizes pro Waldgebiet. Für die automatische Waldüberwachung wurden die Veränderungen
  der Indizes berechnet, sowie starke Veränderungen (z.B. durch Waldrodungen) mittels eines
  Grenzwerts als solche gekennzeichnet.%
}%


%%%%%%%%%%%%%%%%%%%%%%%%%%%%%%%%%%%%%%%%%%%

% time: Wednesday 17:05
% URL: https://pretalx.com/fossgis2023/talk/fossgis2024-38973-dipas-digitale-brgerbeteiligung-mit-open-source-open-data/

%

\noindent\abstractHSdrei{%
  Daniel Bockelmann, Mateusz Lendziński%
}{%
  DIPAS~-- Digitale Bürgerbeteiligung mit Open Source \& Open Data%
}{%
}{%
  Das Digitale Partizipationssystem DIPAS schlägt eine Brücke zwischen Bürgerbeteiligung Online und
  vor Ort und setzt dabei mithilfe interaktiver Touchtables auf offene Geodaten und Open-Source
  Technologien. Im Vortrag werden das Tool sowie die DIPAS Anwender Community vorgestellt.%
}%


%%%%%%%%%%%%%%%%%%%%%%%%%%%%%%%%%%%%%%%%%%%

% time: Wednesday 17:05
% URL: https://pretalx.com/fossgis2023/talk/fossgis2024-39031-geodaten-mit-duckdb-verarbeiten/

%

\noindent\abstractHSvier{%
  Nikolai Janakiev, Jakob Miksch%
}{%
  Geodaten mit DuckDB verarbeiten%
}{%
}{%
  DuckDB hat sich als leichtgewichtiges Werkzeug für Datananalysen aller Art in der Data Science
  Community etabliert. Mittlerweile gibt es auch eine offizielle Erweiterung die mit Geodaten
  arbeiten kann. In diesem Vortrag stellen wir die grundlegenden Funktionen vor.%
}%


%%%%%%%%%%%%%%%%%%%%%%%%%%%%%%%%%%%%%%%%%%%

% time: Wednesday 17:40
% URL: https://pretalx.com/fossgis2023/talk/fossgis2024-38891-nachhaltige-beschaffung-mit-blick-auf-open-source/

%
\newTimeslot{17:40}
\noindent\abstractHSeins{%
  Torsten Friebe, David Arndt, Florian Micklich, Miriam Seyffarth%
}{%
  Nachhaltige Beschaffung mit Blick auf Open Source%
}{%
}{%
  Seit Juni 2021 beschäftigt sich die Arbeitsgruppe ["Öffentliche Ausschreibungen mit FOSS"' des
  FOSSGIS e.V.](https://www.fossgis.de/wiki/AG\_oeffentl\_Ausschreibungen\_FOSS) mit dem Thema der
  Beschaffung und Vergabe von IT-Lösungen auf Basis von FOSS. In dieser Dialogrunde wollen wir mit
  Vertreter:innen aus den verschiedenen Bereichen der digitalen Community Fragen zur nachhaltigen
  Beschaffung mit Blick auf Open Source und Digitalisierung diskutieren.%
}%


%%%%%%%%%%%%%%%%%%%%%%%%%%%%%%%%%%%%%%%%%%%

% time: Wednesday 17:40
% URL: https://pretalx.com/fossgis2023/talk/fossgis2024-38924-open-source-and-web-based-geoai-tool-for-transparent-forest-fire-prediction/

%

\noindent\abstractHSzwei{%
  Qasem Safariallahkheili, Sebastian Meier%
}{%
  Open Source and Web-Based GeoAI tool for Transparent Forest Fire Prediction%
}{%
}{%
  Utilizing open geospatial data and AI, we are trying to predict forest fire susceptibility in
  Brandenburg. This case study showcases full-stack web-GIS, emphasizing user interaction and
  transparent AI through open-source tools like GEE, GDAL, Python, Maplibre, and TiTiler.%
}%


%%%%%%%%%%%%%%%%%%%%%%%%%%%%%%%%%%%%%%%%%%%

% time: Wednesday 17:40
% URL: https://pretalx.com/fossgis2023/talk/fossgis2024-38992-ein-webgis-zur-ffentlichkeitsbeteiligung-in-planungsverfahren-des-netzausbaus/

%
\newpage
\noindent\abstractHSdrei{%
  Kim-Jana Stückemann%
}{%
  Ein WebGIS zur Öffentlichkeitsbeteiligung in Planungsverfahren des Netzausbaus%
}{%
}{%
  Zur Partizipation von Trägern öffentlicher Belange, Interessenverbänden und Bürger:innen wurde im
  Rahmen des Netzverstärkungsprojekts Höchstspannungsleitung Elsfleth/West~-- Ganderkesee –
  Umspannwerk HUCH (Berne/Lemwerder/Ganderkesee), getragen durch die TenneT TSO GmbH, ein Open
  Source WebGIS basierend auf OpenLayers und GeoServer entwickelt, mithilfe dessen die beteiligten
  Akteur:innen aktuelle Planungsinformationen erlangen sowie ihr lokales Wissen mit der
  Projektträgerin teilen können.%
}%


%%%%%%%%%%%%%%%%%%%%%%%%%%%%%%%%%%%%%%%%%%%

% time: Wednesday 17:40
% URL: https://pretalx.com/fossgis2023/talk/fossgis2024-39049-the-sparql-unicorn-ontology-documentation-exposing-rdf-geodata-using-static-geoapis/

%

\noindent\abstractHSvier{%
  Florian Thiery, Timo Homburg%
}{%
  The SPARQL Unicorn Ontology documentation: Exposing RDF geodata using static GeoAPIs%
}{%
}{%
  We introduce the ontology documentation feature of the SPARQLing Unicorn QGIS Plugin, allowing the
  conversion of RDF data dumps to static HTML deployments with static versions of well-known APIs
  such as OGC API Features. Contents of RDF data are deployed interoperably accessible for many
  research communities. Our talk shows the tool's motivation, conversion process with test datasets,
  discusses limitations, standardization issues and future developments of this approach in general.%
}%


%%%%%%%%%%%%%%%%%%%%%%%%%%%%%%%%%%%%%%%%%%%

% time: Wednesday 18:15
% URL: https://pretalx.com/fossgis2023/talk/fossgis2024-38975-qgis-plugin-what-s-the-impact-of-my-installed-dam-on-the-vegetation-around-it-/

%
\newTimeslot{18:15}
\noindent\abstractHSzwei{%
  Berit Mohr%
}{%
  QGIS Plugin: What's the impact of my installed dam on the vegetation around it?%
}{%
}{%
  Woher weiß ich, dass die Vegetation sich regeneriert augrund von meines installierten Dammes? In
  diesem Vortrag werde ich den QGIS Plugin vorstellen, den wir im Rahmen eines Projektes in
  Äthiopien für die Gesellschaft für Internationale Zusammenarbeit (GIZ) entwickelt haben.%
}%


%%%%%%%%%%%%%%%%%%%%%%%%%%%%%%%%%%%%%%%%%%%

% time: Wednesday 18:15
% URL: https://pretalx.com/fossgis2023/talk/fossgis2024-38901-inspire2gpkg/

%

\noindent\abstractHSvier{%
  Armin Retterath%
}{%
  INSPIRE2GPKG%
}{%
}{%
  Mit einer einfachen Python-Bibliothek lassen sich Raster- und Vektordaten automatisiert aus einer
  INSPIRE-kompatiblen GDI extrahieren und in einem Geopackage speichern. Im Vortrag werden die in
  der Software umgesetzten Prozesse und Prinzipien erläutert, und die Funktionsweise wird anhand
  praktischer Beispiele vorgestellt. Das Verfahren zeigt eindrucksvoll welches bisher noch
  ungenutzte Potential in den auf INSPIRE basierenden Infrastrukturen sowie im GPKG-Format steckt.%
}%

\newpage
%%%%%%%%%%%%%%%%%%%%%%%%%%%%%%%%%%%%%%%%%%%

% time: Wednesday 18:15
% URL: https://pretalx.com/fossgis2023/talk/fossgis2024-38473-ol-describe-map-mehr-web-accessibility-fr-openlayers-karten/

%

\noindent\abstractHSdrei{%
  Marc Jansen%
}{%
  ol-describe-map: Mehr Web-Accessibility für OpenLayers Karten%
}{%
}{%
  In diesem Lightning Talk möchte ich die ol-describe-map-Bibliothek vorstellen und erläutern, wie
  sie zur Verbesserung der Accessibility in Webkarten beitragen kann. Die Bibliothek ist zwar noch
  jung \& sicher auch fern von feature-complete, aber sie hat das Potenzial, den Zugang zu Webkarten
  für alle Nutzer:innen zu erleichtern. Ich werde die Grundfunktionalitäten vorstellen, zeigen wie
  man speziellere Anforderungen umsetzen kann und einen kleinen Ausblick geben.%
}%



%%%%%%%%%%%%%%%%%%%%%%%%%%%%%%%%%%%%%%%%%%%

% time: Wednesday 18:20
% URL: https://pretalx.com/fossgis2023/talk/fossgis2024-38600-gis-mit-go/

%
\newSmallTimeslot{18:20}
\noindent\abstractHSdrei{%
  Jakob Miksch%
}{%
  GIS mit Go%
}{%
}{%
  Die Programmiersprache Go hat sich in verschiedenen IT-Bereichen etabliert. Dieser Lightning Talk
  bietet einen kurzen Überblick über Go mit einem Fokus auf das bestehende Ökosystem um Geodaten zu
  verarbeiten.
  Go ist bekannt für Geschwindigkeit und Zugänglichkeit. Zahlreiche GIS-Projekte wie pg\_featureserv,
  pg\_tileserv und tegola nutzen bereits Go. Dieser Vortrag präsentiert weitere Tools und
  Bibliotheken in dieser Sprache.%
}%


%%%%%%%%%%%%%%%%%%%%%%%%%%%%%%%%%%%%%%%%%%%

% time: Wednesday 18:25
% URL: https://pretalx.com/fossgis2023/talk/fossgis2024-38819-actinia-wachsen-bltter-geoprozessierung-und-visualisierung-mit-leafmap-in-jupyter/

%
\newSmallTimeslot{18:25}
\noindent\abstractHSdrei{%
  Markus Neteler, Carmen Tawalika%
}{%
  actinia wachsen Blätter: Geoprozessierung und Visualisierung mit Leafmap in Jupyter%
}{%
}{%
  [Leafmap](https://leafmap.org/) ist ein freies und quelloffenes Python-Paket für interaktives
  Mapping und räumliche Analyse von Geodaten in einer Jupyter-Umgebung. Leafmap bietet interaktive
  Werkzeuge, mit denen Geodaten einfach in eine Karte geladen werden können. Wir haben die
  Cloud-Plattform für Geoverarbeitung [actinia](https://actinia-org.github.io/) mit Leafmap
  gekoppelt, um Berechnungen aus actinia direkt anzeigen zu können. Im Lightning Talk wird dies an
  einem Beispiel demonstriert.%
}%


%%%%%%%%%%%%%%%%%%%%%%%%%%%%%%%%%%%%%%%%%%%

% time: Wednesday 18:30
% URL: https://pretalx.com/fossgis2023/talk/fossgis2024-38756-remote-plugin-installer/

%
\newSmallTimeslot{18:30}
\noindent\abstractHSdrei{%
  Frida Kessler%
}{%
  Remote Plugin Installer%
}{%
}{%
  QGIS Plugins via POST Request installieren und damit den Entwicklungsworkflow beschleunigen. Quasi
  Remote Code Execution für QGIS.%
}%


%%%%%%%%%%%%%%%%%%%%%%%%%%%%%%%%%%%%%%%%%%%
